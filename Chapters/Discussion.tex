\chapter{Discussion}
\protect\label{chapter:discussion}
\lhead{Chapter 5. \emph{Discussion}}

In consideration of the methods of processing the actual observational data, the consistency between the various results
from \asas, supported by the {\hst} results, cannot be ignored. Not only do the three Lomb-Scargle routines tried yield
identical results but the result of 82.6 $\pm$ 0.1 days for the period of {\prox} is completely consistent. An
explanation for the secondary peak of 106.3 $\pm$ 0.1 days as being a ``beat'' between the rotation period and an Earth
year is also available which {\Firstp} believe further confirms the likelihood of the rotation period of {\prox} being
at or close to 82.6 days, especially when this is not observed in the {\hst} results, which are far less constrained by
the time of year when observations of {\prox} may be made. It is therefore feasible to use the 82.6 day period, together
with the 106.4 day period as a ``benchmark'' to assess the accuracy of the other methods of obtaining the rotation
period.

In this context the results obtained using these methods involving analysis of the {\ha} peak of the spectra are rather
mixed in their success. The three Lomb-Scargle routines tried give radically different results in most cases. The
treatment of the data, in terms of taking residuals, clipping of extreme data and binning sometimes improve and
sometimes worsen the accuracy of the returned value or increase or reduce the likelihood of a false positive. Sometimes
half of the period is returned.

The Peak Ratio does appear to perform better than the other methods, returning close to 82.6 days more often than the
other two methods. The {\ha} Index method and the Equivalent Width method of analysing the peaks give the most similar
results, in many cases identical to within 0.1 day, or possibly with some peaks interchanged. It has to be noted that
the {\ha} Index method effectively normalises the data itself, whereas the Equivalent Width method uses data which has
been normalised already. This perhaps gives added confidence in the applicability of both methods.  If the search space
is restricted to periods in the region of 82.6 days, then it is clear that this value, or within a few percent of it,
can be extracted using all the search methods, but standing alone, they cannot be considered to be a reliable
periodicity measure for \prox, however they may well fare better for for other stars.

It has proved possible to reproduce the 116.6 days of \citet[Table 3]{suarezmascareno15} in both the treatments of
Equivalent Width and {\ha} Index and in some of the other variants of the handling of those, as shown in Table
\ref{table:peak5compall}, together periodograms taken from the skewness and kurtosis. However, this peak is the fifth
strongest in the periodograms if the period search is extended to 40 days and disappears if any kind of selection or
binning is made from the data or if additional data from 2016 is included. In addition, it was noticed that some of the
modelling results which failed to give the expected period (see Section \ref{section:addflares}) also gave periods close
to 116.6 days from the same observation times as in the {\harps} data. From this analysis, {\Firstp} would have to
discount this as a false positive, probably an artefact of the observation times and confirm the previously-assigned
rotation period of the order of 82.5 days in \citet{benedict98} and confirmed elsewhere, including the photometric
{\asas} and {\hst} data discussed in this paper and some of the results from the spectroscopic data.

It appears clear that there is a contamination of the {\ha} line of {\prox} from another periodic effect or effects
specific to {\ha} which seriously compromise the utility of several measures of periodicity from the study of this line,
although the Peak Ratio method would appear to fare better, especially when observations affected by the possible flare
activity are masked out.
%{\FirstP} briefly investigated the possibility of extracting periodicity from an absorption line identified
%as a TiO transition
%and highlighted in the second panel of Fig. \ref{fig:harpsfirstha}
%but found the Periodogram results scarcely different from that from that from {\ha} in terms of identifying any strong
%signal.
% As is clear from Fig. \ref{fig:harpsfirstha} this line is much smaller than {\ha} and the corresponding
% proportionate error is much greater.
A future line of investigation might be worth considering by combining fluxes from various lines in various spectra
orders to evaluate the Composite Spectral Index referred to in \citet{hall99} and \citet{hall00}.

Turning to the modelling results, whilst {\Firstp} were able to reproduce the variations in Equivalent Width seen in
{\prox} using the DoTS model and show that the measurement of rotation period from periodicity has validity down to
quite a poor SNR and despite the presence of significant flares, notwithstanding extensive experimentation, including
relatively extreme values for the various parameters for limb-darkening and contrast or extreme distributions of plage,
it proved impossible to approach even remotely the observed variations in Peak Ratio observed either in the {\harps} or
{\uves} data and listed in Table \ref{table:ewtabfirst}.

The variations in the two \horn s that are purely due to Doppler shift from the rotational velocity are not large; with
a radius of 0.141 Solar \citep{demory09} and assuming a period of the order of 80 days the rotational velocity is at
most 90 m/s yielding a Doppler shift of at most 0.003{\AA} in the {\ha} line between the extremes of the disk and the
centre. Clearly the DoTS modelling of static plage is not adequate to emulate either the asymmetry or the variations in the
\horn s in the {\ha} line profile as illustrated in Fig. \ref{fig:harpsfirstha}, as this rotational velocity is far too
low for a Doppler shift to produce them from plage distributions as illustrated in Table \ref{table:modelcomp}.

Trying to reproduce all the observed effects by flux profiles of plage regions in the chromosphere is undoubtedly
over-simplistic. Moreover the combinations of limb darkening laws and especially the contrast refinements referred to in
Section \ref{chapter:modelling}, whilst tuned to give reasonable contrast in the model, and whilst a reasonable starting
point, are partly derived for the Sun and probably not appropriate for the very different situation represented by
\prox.

Despite the inadequacies of the model, a reasonably accurate periodicity from the {\ha} Equivalent Width is obtainable
even down to SNR levels in the teens, well below the noise levels for {\harps} and {\uves} which are mostly over
100. The model and these techniques may well be suitable for stars with a much lower rotation period and with rotational
velocity much greater than that of {\prox} which has one of less than 90 m/s and provided that any other activity on the
star in question does not greatly affect the {\ha} line in proportion. It is hard to give more than a qualitative view
of the validity of the Peak Ratio method with this model as applied to \prox, although {\Firstp} did attempt to obtain
periods from the Peak Ratios in the model, despite the negligible variations, obtaining results approaching (and in a
few cases better than) the accuracy of the Equivalent Width results.

{\FirstP} considered various scenarios which might apply to cause the variations in the Peak Ratio. If {\prox} and stars
exhibiting similar behaviour are to be adequately modelled, the model used will have to be significantly extended to
cater for a wider set of effects. One aspect would be the possibility of differential rotation as discussed in as
discussed for CJ1243 in \citet{davenport15}. Ruled out in \citet{barnes05}.
% This would probably not apply to {\prox} as it has a mass of 0.123 Solar \citep{segransan03} and a
% radius of 0.141 Solar \citep{demory09} this means the average density of {\prox} is about 44 times Solar which would
% make this seem unlikely. Moreover
The results from {\hst} and {\asas} in Section \ref{chapter:photometry} appear to rule out differential rotation periods.
Also such a model ought to cater the possibility of Ellerman Bombs as described by Ellerman (1917) and recently
discussed in \citet{rutten13} and \citet{vissers15} and \citet{rutten16} as particularly affecting the {\ha} line.
% These however were all treatments of the Sun. Again, this probably would not apply to {\prox} as they relate to much
% higher temperatures, in the range 10,000K to 20,000K, which would be very much less frequent with {\prox} and are
% explicitly stated to be a photospheric phenomenon.
Another feature to be catered for in a more comprehensive model would be the possibility of prominence and plage
activity as discussed by \citet{eibe98} for RE~1816+541. However this is a rapidly-rotating star, of the order of 12
hours, rather than the 82 days of \prox. Also the literature such as \citet{skelly09}, building on earlier work such as
\citet{donati99}, \citet{dunstone06} and \citet{colliercameron89} discusses the possibility of slingshot
prominences. These however focus on K-dwarfs and appear to be associated with strong magnetic fields which would not be
expected with a slow rotator such as \prox, although possibly the dynamo discussed in \citet{yadav16} may provide a clue
to this.

{\FirstP} finally noted the work on chromospheres in the review of \citet[Section VII]{linsky80} considering systematic
chromospheric flow patterns. Linsky mostly considers Ca lines, but that review cites \citet{athay70} which particularly
considers {\ha} and also \citet{athay70a}. These papers and others argue that vertically propagating shock waves can
account for the red and blue asymmetries observed in various lines. In \citet{shkolnik03}, the authors build on this
work to suggest that planets would induce such chromospheric activity on HD 179949 (an F8V star). The presence of the
red and blue asymmetries in the {\ha} line on {\prox} as {\Firstp} observed does suggest this as a possible line of
enquiry to build into new models. Somewhat similar to this is the discussion of line asymmetry of H, K and Ca lines in
\citet{rauscher06}, however these were for earlier type of stars, K7 to M5, than \prox, for which {\ha} is in
absorption. The \horn s shown in fig. 2 in that paper have the red {\horn} smaller than the blue {\horn} as opposed to
{\prox} as shown in Fig. \ref{fig:harpsfirstha}, for which it is in most cases the other way round. In section 4.2 of
that paper the authors suggest that line asymmetry is caused by a slowly-decelerating motion toward the observer which
does not fall back ballistically away from the observer which would therefore be blue-shifted and hence enhance the blue
\horn. On the other hand with \prox, it might be a similar effect is taking place but the self-absorption portion is
blue-shifted which would be inverted to resemble a red shift.

Future directions for {\prox} would therefore be to develop models to simulate some of these effects and attempt to
relate them to the observed variations, particularly in the Peak Ratio. As for the measurements based on the {\ha} line
and probably other lines, whilst {\Firstp} consider that they are currently less powerful than photometry for {\prox} in
isolation, they may well be useful for other stars with a smaller rotation period whose Doppler variations in the
morphology of the {\ha} and other lines dominate the effects.


\chapter{Discussion}
\protect\label{chapter:discussion}
\lhead{Chapter \ref{chapter:discussion} \emph{Discussion}}

It is clear that the period of 82.6 $\pm$ 0.1 days given by the photometric results for {\asas} and confirmed by {\hst}
in Chapter \ref{chapter:photometry} must be the rotation period of \prox, in line with \citet{benedict98} and confirmed
by \citet{kiraga07}. There is a near-zero FAP value against 82.6 days and all the Lomb-Scargle routines which were tried
gave exactly the same result with identical periodograms (apart from allowances for scaling of the power). The
experiments in Section \ref{section:asasfap} with taking subsets of the data and noting the changes in FAP and the
standard deviation of the error, with only limited extrapolation of the chart in Fig. \ref{fig:asasprop}, give
confidence in assigning the error bar on this period as being no more than 0.1 days.

It was not possible to obtain as clear-cut results from spectroscopic methods involving analysis of the {\ha} peak of
the {\prox} spectra, in terms of returning the period at all, obtaining a clear-cut topmost peak in the periodograms or
obtaining a reasonable error bar. As can be seen in the summary in Table \ref{table:perftable} or the full results in
Appendix \ref{chapter:pgramdetail}, the Peak Ratio method appears to perform significantly better more often than the
Equivalent Width method, with Skewness and Kurtosis methods somewhere in between and more consistent. They are of the
same kind of order as the performance illustrated in Fig. \ref{fig:asasprop}, for the same number of observations, but
clearly cannot be used standing alone to compute the rotation period. Clipping high equivalent width observations, or
binning to various periods can improve some results, but not in any consistent way.

There is also a strong peak of 106.3 $\pm$ 0.1 days on the {\asas} results and seen in some of the spectroscopic results
and the modelling, but not seen in the {\hst} results. This period would appear to be a ``beat'' between the rotation
period and an Earth year, which would not affect the {\hst} results, which are far less constrained by the time of year.

It has proved possible to reproduce the 116.6 days of \citet[Table 3]{suarezmascareno15} in both the treatments of
Equivalent Width and {\ha} Index and in some of the other variants of the handling of those, together with periodograms
taken from the skewness and kurtosis. However, this peak is the fifth strongest in the periodograms if the period search
is extended to 40 days and disappears if any kind of selection or binning is made from the data or if additional data
from 2016 is included. In addition, it was noticed that some of the modelling results which failed to give the expected
period (see Section \ref{section:addflares}) also gave periods close to 116.6 days from the same observation times as in
the {\harps} data. From this analysis, this would have to be discounted as a false positive, probably an artefact of the
observation times.

Limiting the portion of the spectrum to just the {\ha} line of \prox, even with the instrumental stability of {\harps}
was proven to be less useful than {\asas} ground-based photometry for the recovery of period. A future line of
investigation which might be worth considering is that of combining fluxes from various magnetic/activity sensitive lines
in various spectral orders to re-evaluate the Composite Spectral Index referred to in \citet{hall99} and
\citet{hall00}. This was briefly explored in the context of the examination of the TiO line in Section
{section:harpsper}, but little benefit was achieved.

It was possible to reproduce the variations in Equivalent Width seen in {\prox} using the DoTS model and show that the
recovery of the rotation period has validity. However, even with extensive experimentation, including relatively extreme
values for the various parameters for limb-darkening and contrast or extreme distributions of plage, it was not possible
to model the observed variations in Peak Ratio found either in the {\uves} or {\harps} data. It is clear that the
variations in the two \horn s that are purely due to Doppler shift from the rotational velocity are not large; with a
radius of 0.141 Solar \citep{demory09} and assuming a period of the order of 80 days the rotational velocity is at most
90 m/s yielding a Doppler shift of at most 0.003{\AA} in the {\ha} line between the extremes of the disk and the centre,
far too low to reproduce the variations in the {\ha} line profile as as illustrated in Fig. \ref{fig:harpsfirstha} for
which the Peak Ratios were calculated as 0.997 $ \pm $ 0.018 (see Table \ref{table:ewtabfirst}). On the other hand, the
best standard deviation on a Peak Ratio close to 1.0 which could be obtained from the models was $2{\times}10^{-5}$ or
$2{\times}10^{-4}$ with very extreme plage distributions. This was not surprising due to the lack of Doppler broadening
of the line profile.

It is clear that the 2D model of static plage and spots supported by DoTS cannot reproduce the observed variations in
the Peak Ratios. Likewise it cannot reproduce the range of phenomena which adversely affects obtaining periodicity from
the Equivalent Widths. This points to the need for a 3D model including vertical processes to properly understand the
behaviour of \prox. In \citet{mohanty02} and \citet{mohanty03}, where the activity of late \rdwarf s is found to be less
closely tied to the rotation period for earlier type stars the authors propose a ``turbulent dynamo'' as the source of
the activity, for which a 3D model is required.

%A variety of phenomena in the literature were considered for this \paperorthesis. Also such a model ought to cater the
%possibility of Ellerman Bombs as described by Ellerman (1917) and recently discussed in \citet{rutten13} and
%\citet{vissers15} and \citet{rutten16} as particularly affecting the {\ha} line.
% These however were all treatments of the Sun. Again, this probably would not apply to {\prox} as they relate to much
% higher temperatures, in the range 10,000K to 20,000K, which would be very much less frequent with {\prox} and are
% explicitly stated to be a photospheric phenomenon.
%Another feature to be catered for in a more comprehensive model would be the possibility of prominence and plage
%activity as discussed by \citet{eibe98} for RE~1816+541. However this is a rapidly-rotating star, of the order of 12
%hours, rather than the 82 days of \prox. Also the literature such as \citet{skelly09}, building on earlier work such as
%\citet{donati99}, \citet{dunstone06} and \citet{colliercameron89} discusses the possibility of slingshot
%prominences. These however focus on K-dwarfs and appear to be associated with strong magnetic fields which would not be
%expected with a slow rotator such as \prox, although possibly the dynamo discussed in \citet{yadav16} may provide a clue
%to this.

In \citet{leenaarts12} a successful 3D MHD simulation was performed for the Sun. Also work on seismic shock waves has
been undertaken for example in \citet{donea06}. Similar conclusions are reached by \citet{rauscher06} as an explanation
for H, K and Ca line asymmetry, however these were for earlier type of stars than \prox, of type K7 to M5, for which
{\ha} is in absorption. The \horn s shown in fig. 2 in that paper have the red {\horn} smaller than the blue {\horn} as
opposed to {\prox} as shown in Fig. \ref{fig:harpsfirstha}, for which it is in most cases the other way round. The
authors suggest that this effect is caused by slowly-decelerating motion toward the observer which does not fall back
ballistically away from the observer which would therefore be blue-shifted and hence enhance the blue \horn. With \prox,
it might be a similar effect is taking place but the self-absorption portion is blue-shifted which would be inverted to
resemble a red shift.

%{\FirstP} finally noted the work on chromospheres in the review of \citet[Section VII]{linsky80} considering systematic
%chromospheric flow patterns. Linsky mostly considers Ca lines, but that review cites \citet{athay70} which particularly
%considers {\ha} and also \citet{athay70a}. These papers and others argue that vertically propagating shock waves can
%account for the red and blue asymmetries observed in various lines. In \citet{shkolnik03}, the authors build on this
%work to suggest that planets would induce such chromospheric activity on HD 179949 (an F8V star). The presence of the
%red and blue asymmetries in the {\ha} line on {\prox} as {\Firstp} observed does suggest this as a possible line of
%enquiry to build into new models. Somewhat similar to this is the discussion of line asymmetry of H, K and Ca lines in
%\citet{rauscher06}. In section 4.2 of
%that paper the authors suggest that line asymmetry is caused by a slowly-decelerating motion toward the observer which
%does not fall back ballistically away from the observer which would therefore be blue-shifted and hence enhance the blue
%\horn. On the other hand with \prox, it might be a similar effect is taking place but the self-absorption portion is
%blue-shifted which would be inverted to resemble a red shift.

%Future directions for {\prox} would therefore be to develop models to simulate some of these effects and attempt to
%relate them to the observed variations, particularly in the Peak Ratio. As for the measurements based on the {\ha} line
%and probably other lines, whilst {\Firstp} consider that they are currently less powerful than photometry for {\prox} in
%isolation, they may well be useful for other stars with a smaller rotation period whose Doppler variations in the
%morphology of the {\ha} and other lines dominate the effects.


\chapter{Discussion}
\protect\label{chapter:discussion}
\lhead{Chapter \ref{chapter:discussion} \emph{Discussion}}

It is clear that the period of 82.6 $\pm$ 0.1 days given by the photometric results for {\asas} and confirmed by {\hst}
in Chapter \ref{chapter:photometry} must be the rotation period of \prox, in line with \citet{benedict98} and confirmed
by \citet{kiraga07}. \examrevision{The sub-harmonic of 41.3 days was checked for, and indeed appeared in some of the
  spectroscopic results, but this was not nearly as strong. It was a similar story for double the period of around 165.2
  days.} There was a near-zero FAP value against 82.6 days and all the Lomb-Scargle routines which were tried gave
exactly the same result with identical periodograms (apart from allowances for scaling of the power which differed
between the routines). The experiments in Section \ref{section:asasfap} with taking subsets of the data and noting the
changes in FAP and the standard deviation of the error, with only limited extrapolation of the chart in
Fig. \ref{fig:asasprop}, give confidence in assigning the error bar on this period as being no more than 0.1 days.

It was not possible to obtain as clear-cut results from spectroscopic methods involving analysis of the {\ha} peak of
the {\prox} spectra, in terms of returning the period at all, obtaining a clear-cut topmost peak in the periodograms or
obtaining a reasonable error bar. As can be seen in the summary in Section \ref{section:summspec}, the peak ratio method
appears to perform significantly better, in terms of the error bar and rate of recovery of the correct period, than the
equivalent width method, \examrevision{with skewness and kurtosis methods of possible value, but rather worse in terms of
  accuracy}. The results are of the same kind of order as the performance illustrated in Fig. \ref{fig:asasprop}, for
similar numbers of observations, but clearly cannot be used, standing alone, to compute the rotation period. Clipping
high equivalent width observations, or binning to various periods can improve some results, but not in any consistent
way and only seem to have an effect on equivalent width and peak ratio measurements, not with skewness and kurtosis
measurements, to which these made little difference.

There is also a strong peak of 106.3 $\pm$ 0.1 days on the {\asas} results and seen in some of the spectroscopic results
and the modelling, but not seen in the {\hst} results. This period would appear to be a ``beat'' between the rotation
period and an Earth year, which would not affect the {\hst} results, which are far less constrained by the time of
year. There is also a strong peak of 77.8 days on the {\hst} results which is not seen in the other results. Possibly
this is an alias or similar ``beat'' induced by the {\hst} observation times.

It has proved possible to reproduce the 116.6 days of \citet[Table 3]{suarezmascareno15} in both the treatments of
equivalent width and {\ha} Index and in some of the other variants of the handling of those, together with periodograms
taken from the skewness and kurtosis. However, this peak is the fifth strongest in the periodograms if the period search
is extended to 40 days and disappears if any kind of selection or binning is made from the data or if additional data
from 2016 is included. In addition, it was noticed that some of the modelling results (see Section
\ref{section:addflares}) also gave periods close to 116.6 days from the same observation times as in the {\harps}
data. From this analysis, this would clearly have to be discounted as a false positive, almost certainly an artefact of
the observation times.

Limiting the portion of the spectrum examined to just the {\ha} line of \prox, even with the instrumental stability of
{\harps} was proven to be less useful than {\asas} ground-based photometry for the recovery of period. \examrevision{As
  a possible future line of investigation to be considered for some stars is that of combining fluxes from various
  magnetic/activity sensitive lines in various spectral orders to re-evaluate the Composite Spectral Index referred to
  in \citet{hall99} and \citet{hall00}, but this would be unlikely to be effective for {\prox} with its low rotational
  modulation}. This was very briefly explored in the context of the examination of the TiO line and the He-6678 lines
described in Appendix \ref{chapter:tioline}, \examrevision{with no encouraging results at all, confirming the likely
  in-applicability of such a technique to \prox}.

It was possible to reproduce the variations in equivalent width seen in {\prox} using the DoTS model and show that the
recovery of the rotation period has validity. However, even with extensive experimentation, including relatively extreme
values for the various parameters for limb-darkening and contrast or extreme distributions of plage, it was not possible
to model the observed variations in peak ratio found either in the {\uves} or {\harps} data with a symmetric flux
profile. It is clear that the variations in the two \horn s that are purely due to Doppler shift from the rotational
velocity are not large; with a radius of 0.141 Solar \citep{demory09} and assuming a period of the order of 80 days the
rotational velocity is at most 90 m/s yielding a Doppler shift of at most 0.003{\AA} in the {\ha} line between the
extremes of the disk and the centre, far too low to reproduce the variations in the {\ha} line profile as as illustrated
in Fig. \ref{fig:harps1} for which the peak ratios were calculated as 0.997 $ \pm $ 0.018 (see Table
\ref{table:ewtabfirst}). The best standard deviation on a peak ratio close to 1.0 which could be obtained from the
models was $2{\times}10^{-5}$ or $2{\times}10^{-4}$ with very extreme plage distributions. This was not surprising due
to the lack of Doppler broadening of the line profile.

What was clear from the models was the way in which period recovery was adversely affected by the inclination and
improved as the rotation period was decreased, this is demonstrated by the results obtained in Section
\ref{section:addflares} and samples shown in Fig. \ref{fig:noiseresults} and Fig. \ref{fig:noiseresults60}. This at
least suggests that deducing the rotation period by photometric measurements may become increasingly feasible for faster
rotating stars. \examrevision{The spot and plage configurations are likely to change less between successive rotations
  with a faster rotating star, so this will also increase the probable accuracy of calculations relying upon these.}

It is probably naive to expect the plage configurations to be comparatively unchanged over such a long rotation period
as that for {\prox} and basing the models on this assumption is unlikely to reflect the observed variations
correctly. More importantly, it is also clear that the model of static plage and spots supported by DoTS cannot
accurately reproduce the observed variations in the peak ratios. Likewise it cannot reproduce the range of phenomena which
adversely affects obtaining periodicity from the equivalent widths.

\examrevision{Experimenting with adding the effect of a net vertical velocity to the flux profiles to the DoTS model
  showed some improvement in the variation of the peak ratios, but as noted above, this would be an unphysical velocity
  component and has to be discounted. Also doing this does not adversely effect the equivalent width measure as found in
  the observational data.} This points to the need for a properly-constructed 3D model, \examrevision{taking into
  account activity in the chromosphere and corona other than those close to the photosphere and not possible to model
  with a simple spherical shell}, to properly understand the behaviour of \prox.

\examrevision{In the literature,} similar conclusions are reached by \citet{rauscher06} as an explanation for H, K and Ca
  line asymmetry, however these were for earlier type of stars than \prox, of type K7 to M5, for which {\ha} is in
  absorption. The \horn s shown in fig. 2 in that paper have the red {\horn} smaller than the blue {\horn} as opposed to
  {\prox} as shown in Fig. \ref{fig:harps1}, for which it is in most cases the other way round. The authors suggest that
  this effect is caused by slowly-decelerating motion toward the observer which does not fall back ballistically away
  from the observer which would therefore be blue-shifted and hence enhance the blue \horn. With \prox, it might be a
  similar effect is taking place but the self-absorption portion is blue-shifted which would be inverted to resemble a
  red shift.

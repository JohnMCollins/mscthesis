\chapter{Conclusions}
\protect\label{chapter:conclusions}
\lhead{Chapter \ref{chapter:conclusions} \emph{Conclusions}}

This {\paperorthesis} has set out to study measurements of periodicity via measurements of the {\ha} line in particular
and focusing on \prox.

A study of photometric measurements from {\asas} and {\hst} produced clear evidence of a rotation period of 82.6 $\pm$
0.1 days with negligible FAP. This produced a benchmark from which to evaluate spectroscopic measurements including that
of Equivalent Width and the virtually identical {\ha} Index, which were applied to the {\ha} line to demonstrate that
they did not prove able, in terms of reliable recovery of the period at all or with an acceptable error bar, to deliver
a result in isolation, certainly as far as {\prox} is concerned. The measurement of lines presented in this
\paperorthesis, the Peak Ratio, appears to be consistently better than other measurements and is particularly suitable
for {\ha} line profiles such as that for \prox, which have a ``horned'' appearance with two \horn s around a relatively
unchanging central minimum.

A look at simple 2D models enabled the validity of the method to be established for Equivalent Widths, but not for Peak
Ratios, in the face of varying SNR levels and simulated flares. However the models were ``too good'' for Equivalent
Width whilst not good enough for Peak Ratios. The need for a more sophisticated 3D model with vertical motion is clear
if the activity in \prox, probably other late \rdwarf s with lower rotation period but relatively high activity, are to
be understood.

In this process a recent assessment of the rotation period of {\prox} as 116.6 days was discounted as a false positive.



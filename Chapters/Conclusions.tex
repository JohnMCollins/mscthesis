\chapter{Conclusions}
\protect\label{chapter:conclusions}
\lhead{Chapter 5. \emph{Conclusions}}

In this study, {\Firstp} set out to evaluate the measurements of periodicity obtainable from the study of the {\ha} line
of {\prox} and also attempt to compare it with results from {\Firstposs} simple 3D model, adding the Peak Ratio study to
the existing methods involving {\ha} Index and Equivalent Width etc and compared results of taking subsets of the data,
taking residuals to filter out invariant data. {\FirstP} compared these with the photometric results from {\asas} and {\hst}.

{\FirstP} found that the methods involving study of {\ha} were not sufficiently reliable, at least as far as {\prox} is
concerned, for them to be of use for that star, although the Peak Ratio method introduced here has promise. Nonetheless,
increasing confidence in previous calculations of the rotation period of {\prox} of between 82.5 and 83.0 days emerged
during this process and a recent derivation of 116.6 days using the {\ha} Index was revealed as an almost certainly
incorrect false positive.

{\FirstP} have also attempted to model \prox, but with only very limited success. Despite achieving similar variations
in the Equivalent Width, {\Firstp} were unable to simulate the Peak Ratio variation, due to the low rotational velocity
of {\prox} and consequent minimal effect on the line profile due to Doppler alone. It is clear that future work needs to
be done to consider a wider range of phenomena in order to model {\prox} more accurately.

%This technique used was developed for situations and in particular \prox, where the {\ha} line is in emission and takes
%a noticeable ``horned'' shape where it is easy to pick out the portions which vary noticeably and are discernible by eye
%using an interactive tool. Possible future developments would be to automate the selection of the \horn s but this has
%proved quite difficult for \prox, although {\Firstposs} interactive tool is quick and easy to use and not hard to use to
%pick out similar portions from an absorption spectrum.

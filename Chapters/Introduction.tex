\chapter{Introduction}
\protect\label{chapter:introduction}
\lhead{Chapter \ref{chapter:introduction} . \emph{Introduction}}

\section{The importance of the study of {\rdwarf} stars}
\protect\label{section:introrwarfs}

{\rdwarf} stars are particularly important as a field of study. One major reason for this is that they account for over
75\% of the stars within 25 pc of the Sun, as described in \citet{winters15}. Indeed our nearest neighbour, \prox, is an
M5.5V star. Studies such as \citet{vandokkum2010} have argued for a similar proportion of {\rdwarf} stars in galaxies
other than the Milky Way, although that paper concedes the impossibility of observing them individually
\examrevision{as, for example,} a star such as Barnard's star with an absolute magnitude of about 13 would have a K band
magnitude of about 39 at the distance of the Virgo cluster. The stellar populations of other galaxies is a field of
considerable interest in itself, but one well outside the scope of this \paperorthesis.

\examrevision{{\rdwarf} stars are also of great interest in the search for exoplanets. Having lower mass than earlier
  type stars, the radial velocity variations induced by planetary orbits is proportionally greater. In
  \citet{johnson07}, it is seen that the likelihood of planets approaching Earth-size is actually higher than for G or K
  stars and in \citet{barnes12} a search for such planets in the habitable zone of \rdwarf s was undertaken. In both
  \citet{barnes14} and \citet{tuomi14} the likelihood of such planets in the habitable zone of \rdwarf s is discussed
  further. The habitable zone of the much cooler \rdwarf s would obviously be much closer to the stars than the Sun,
  less than 0.1 AU, with complications such as tidal locking of planets and their exposure to flares. Despite these
  drawbacks to habitability, the longevity of the much slower burning \rdwarf s by comparison with earlier spectral type
  stars would enhance the opportunities for biological evolution to take hold at some stage.}

\examrevision{In the search for the periodic signatures of exoplanets, it is obviously necessary to distinguish between
  those and the rotation period of the star itself. The orbital period of exoplanets of particular interest, as being
  potentially in the habitable zone close to an {\rdwarf}, is of the order of around 10 days, shorter than the 24-day
  rotation period of the sun. Also, as observed later in this \paperorthesis, aliases can arise from the combination of
  two or more different periods. This has been the focus of intensive debate in recent papers which have expressed quite
  divergent views regarding whether reported planets have been validly detected by their period such as in
  \citealt{barnes13,robertson14,robertson14a,tuomi13aug,robertson15}.}

\section{Activity and the rotation period}
\protect\label{section:introrotation}

Despite their abundance, many aspects of the activity of \rdwarf s have remained less well characterised
\examrevision{compared to other classes of star}, mainly because of their inherent faintness. Stars become fully
convective \examrevision{below} around M4V and the transition to this regime has been a subject of some study, as for
example the conference proceedings of \citet{stassun11}.

Of particular interest is the study of complex magnetic activity in such stars and consideration of the dynamo systems
which would give rise to this. Authors such as \citet{morin11} argue for two simultaneous dynamos, whereas
\citet{kitchatinov14} discount this and argue for some of the observed activity being due to magnetic field inversions.

In \citet{mohanty03} the authors set out the correlation between the \examrevision{projected} rotational velocity
{\vsini} and the activity in mid-M to L-dwarfs, \examrevision{charting} activity against rotational velocity up to about
12 km/s where a saturation velocity is observed. They also note an increase \examrevision{in} rotation velocities for
later stars and a drop-off in activity for \ldwarf s. This is developed further in \citet{reiners08} and also
\citet{schmidt14}, however the latter disagree with the previously-reported drop-off in activity for \ldwarf s, arguing
that dust affects the measurements of activity, \examrevision{reporting activity in \ldwarf s down to L6}.

In \citet{mohanty02} the relation between activity and rotational velocity is explored, however both in this and in
\citet{mohanty03} the authors state that the correlation between rotation and activity \examrevision{is not as strong in} 
fully convective stars from M3 on and suggest \examrevision{that a ``turbulent dynamo'' is responsible}.

It is clear that the measurement of rotational velocity is of importance for a full understanding of these processes.

\section{The importance of {\ha}}
\protect\label{section:intohalpha}

The {\ha} line is a powerful diagnostic tool in many areas of astronomy. For example, {\ha} luminosity is one of the key
ways of estimating the star formation rates in galaxies (see for example \citet{rosagonzalez02}).
In terms of the study of stars, since the core of {\ha} forms above the photosphere, as shown in \citet{vernazza81}, it
is an important line for studying activity and variability in stellar chromospheres \citep{hall08}. Chromospheric
features such as plage regions and jets are seen well in {\ha} (see for example \citet{kneer10} and \citet{kuridze11}).
The behaviour of the {\ha} line in {\rdwarf} stars is seen as a key diagnostic of activity, as shown in the
previously-cited papers (\citealt{mohanty02, mohanty03, reiners08, schmidt14}) which describe the relationship between
rotation and activity.

The behaviour of the {\ha} line is a potentially important diagnostic because it is sensitive to magnetic activity and
is a strong line usually seen in emission in later {\rdwarf} stars. It is used as the measure of activity in, for
example \citet{mohanty03} and subsequent papers. In \citet{hatzes15} the {\ha} periodicity is explicitly studied in
relation to \examrevision{GL 581 and whether the exoplanet GL 581d is indeed real or is actually an alias, which has
  been debated extensively in \citealt{barnes13,robertson14,tuomi13aug,robertson15a}}.

This {\paperorthesis} investigates whether periodicity can be identified in the morphology of the {\ha} line obtainable
from high-resolution spectra such as those obtained from the \textit{Ultraviolet and Visual Echelle Spectrograph}
({\uves}) at the 8.2m Very Large Telescope (VLT, UT 2) and the \textit{High Accuracy Radial velocity Planet Searcher}
({\harps}) at the ESO La Silla 3.6m telescope. \citet{barnes14} noticed that while a strong transient emission in {\ha}
was observable during flares, \examrevision{it should nevertheless be possible to derive periodicity from the morphology
  of the {\ha} line. On the other hand, it must be said} that in \citet{reiners09}, the authors suggest that Ca, He and
{\ha} lines, being chromospheric emissions, are strongly affected by activity and should be omitted when searching for
radial velocities in active \rdwarf s.

Conscious of these divergent views, this {\paperorthesis} attempts to study the merits of precision measurements using
chromospheric features and in particular {\ha} analysis.

\section{\prox}
\protect\label{section:introprox}

As the nearest star to the solar system at a distance of 1.3 parsecs, {\prox} is a bright M5.5V star with a magnitude of
11.13 in the V-band and is of obvious interest with extensive data sets available. The {\ha} line is always in emission
with a characteristic shape. Thus is it a good place to test the general methodology applicable to late \rdwarf s.

\examrevision{{\prox} is part of a system of three stars together with Alpha Centauri A and B and of higher metallicity
  than the Sun, as described in \citet{linsky04}. This further focuses interest on the search for rocky planets and ones
  to be found in the habitable zones of those stars, with one recently discovered for {\prox} and reported in
  \citet{angladaescude16}.}

Despite the interest and available data, the rotation period of {\prox} is nevertheless uncertain.  Previous studies
have reported periods ranging from the $ 31.5 \pm 1.5 $ days of \citet{guinan96}, the 41.3 days of \citet{benedict93}
and s between 82 and 84 days in \citealt{benedict92,benedict98}.  \citet{kurster99} found that the period is not less
than 50 days, \examrevision{whilst more recently \citet{kiraga07} reported a value of 82.5 days}. \examrevision{All these
  estimates were based on photometric measurements}.  In \citet{cincunegui07} a 447-day activity cycle is reported,
based upon {\ha} measurements. Latterly \citet[Table 3]{suarezmascareno15} reported a value of 116.6 days, again using a
measurement of {\ha}.

{\prox} has quite frequent flares \examrevision{and whilst they obviously disturb the measurements, they enable a study
  to be made of and to what extent they compromise} methods for the calculation of the rotation period.

\section{Methodology}
\protect\label{section:methodology}

The approach taken in this {\paperorthesis} is as follows \examrevision{is similar} to that taken \examrevision{by}
\citet{giguere16} for the K dwarf $\epsilon$ Eridani. In Chapter \ref{chapter:photometry} photometric measurements of
\examrevision{the} periodicity \examrevision{of} {\prox} are examined, as these prove a useful benchmark for
spectroscopic measurements. In Chapter \ref{chapter:proxima} the measurement methods are introduced which are
subsequently used to investigate periodicity from sets of spectra. In Chapter \ref{chapter:modelling} possible models of
{\prox} are studied and attempts \examrevision{are} made to apply the measurement methods to simulated spectra to evaluate their
performance, obtain error estimates and determine how \examrevision{these vary} with poorer signal to noise ratio. Chapter
\ref{chapter:discussion} and \ref{chapter:conclusions} report the discussion and conclusions of this study. Appendices
describe some of the software tools developed and used and list \examrevision{some additional results}.

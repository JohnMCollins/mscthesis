\chapter{Introduction}
\protect\label{chapter:introduction}
\lhead{Chapter \ref{chapter:introduction} . \emph{Introduction}}

The {\ha} line is a powerful diagnostic tool in many areas of astronomy. For example, {\ha} luminosity is one of the key
ways of estimating the star formation rates in galaxies (see for example \citet{rosagonzalez02}.  Since the core of
{\ha} forms above the photosphere \citep{vernazza81}, it is an important line for studying activity and variability in
stellar chromospheres \citep{hall08}. Chromospheric features such as plage regions and jets are seen well in {\ha} ( see
for example \citet{kneer10} and \citet{kuridze11}).

The behaviour of the {\ha} line in {\rdwarf} stars is seen as a key diagnostic of activity. {\rdwarf} stars are
particularly important for a host of reasons; not only are they very numerous, accounting for over 75\% of the stars
within 25 pc in the Solar neighbourhood, as described in \citet{winters15}. The correlation of activity, magnetic fields
and the rotation period has proved of particular importance (e.g. \citet{mohanty02}, \citet{mohanty03},
\citet{reiners08} and \citet{schmidt14}).

Here {\Firstp} investigate whether periodicity can be identified in the morphology of the {\ha} line obtainable from
high-resolution spectra such as those obtained from the \textit{Ultraviolet and Visual Echelle Spectrograph} ({\uves})
at the 8.2m Very Large Telescope (VLT, UT 2) and the \textit{High Accuracy Radial velocity Planet Searcher} ({\harps})
at the ESO La Silla 3.6m telescope.  \citet{barnes14} noticed that while a strong transient emission in {\ha} was
observable during flares, the jitter was limited. Also in \citet{reiners09}, it is observed that that activity does not
greatly affect radial velocity jitter in \rdwarf s. This suggests that it may well be possible to track periodicity with
precision by methods using chromospheric features and in particular {\ha} analysis.

An understanding of periodicity is important for the search and detection of exoplanets. Recent papers have expressed
quite divergent views regarding whether reported planets have been validly detected by their period (see
\citealt{barnes13,robertson14,robertson14a,tuomi13aug,robertson15}). In \citet{hatzes15} the {\ha} periodicity is 
explicitly studied in relation to the much-discussed GL 581 and possible GL 581d.

This study focuses on {\prox} for a number of reasons. As the nearest star to the solar system at 1.3 parsecs it is
obviously of particular interest with extensive data sets available. The {\ha} line is always in emission with a
characteristic shape. Thus is it a good place to test the general methodology applicable to late \rdwarf s.

Previous studies of the rotation period of {\prox} have given a value initially suggested as 31.5 $\pm$ 1.5 days in
\citet{guinan96}, a value of 41.3 days in \citet{benedict93} and as between 82 and 84 days in
\citealt{benedict92,benedict98}, confirmed as being not less than 50 days in \citet{kurster99} and as 82.5 days of
\citet{kiraga07}. Latterly it has been given as 116.6 days in \citet[Table 3]{suarezmascareno15}.  {\prox} has quite
frequent flares and the presence of these flares is useful for determination of whether and to what extent they disturb
methods for the calculation of the rotation period. The previous studies of rotation periods have found variations in
the overall intensity as in \citet{benedict98} and \citet{kiraga07}. In \citet{cincunegui07} is reported a a 447-day
activity cycle.In this paper {\Firstp} make use of archival high resolution spectroscopic observations to study periodic
changes in the detailed morphology of the {\ha} line and evolve additional techniques.

In chapter \ref{chapter:photometry} {\Firstp} discuss photometric measurements of periodicity from \prox, as these prove a
useful benchmark for spectroscopic measurements. In section \ref{chapter:proxima} {\Firstp} introduce the measurement
methods which {\Firstp} study in this paper and subsequently use to investigate periodicity from sets of spectra. In
section \ref{chapter:modelling} {\Firstp} go on to generate possible models of {\prox} and attempt to apply the
measurement methods to simulated spectra to evaluate their performance, obtain an error estimate and how this varies
with poorer signal to noise ratio. Sections \ref{chapter:discussion} and \ref{chapter:conclusions} report {\Firstposs}
discussion and conclusions.


\chapter{Introduction}
\protect\label{chapter:introduction}
\lhead{Chapter \ref{chapter:introduction} . \emph{Introduction}}

\section{The importance of the study of {\rdwarf} stars}
\protect\label{section:introrwarfs}

{\rdwarf} stars are particularly important as a field of study. One major reason for this is that they account for over
75\% of the stars within 25 pc of the Sun, as described in \citet{winters15}. Indeed our nearest neighbour, \prox, is an
M5.5V star. Studies such as \citet{vandokkum2010} have argued for a similar proportion of {\rdwarf} stars in galaxies
other than the Milky Way, although that paper concedes the impossibility of observing them individually as for example a
star such as Barnard's star with an absolute magntiude of about 13 would have a K band magnitude of about 39 at the
distance of the Virgo cluster. The stellar populations of other galaxies is a field of considerable interest in itself,
but one well outside the scope of this \paperorthesis.

Despite their abundance, many aspects of the activity of {\rdwarf} have remained less well characterised, mainly because
of their inherent faintness. Stars become fully convective at around M4V and the transition to this regime has been a
subject of some study, as for example the conference proceedings of \citet{stassun11}.

Of particular interest is the study of complex magnetic activity in such stars and consideration of the dynamo systems
which would give rise to this. Authors such as \citet{morin11} argue for two simultaneous dynamos, whereas
\citet{kitchatinov14} discount this and argue for some of the observed activity being due to magnetic field inversions.

\section{Activity and the rotation period}
\protect\label{section:introrotation}

In \citet{mohanty03} the authors set out the correlation between the rotational velocity $ v sin i $ and the activity in
Mid-M to L-dwarfs and chart activity against rotational velocity up to about 12 km/s where a saturation velocity is
observed. They also note an increase of rotation velocities for later stars and a drop-off in activity for
\ldwarf s. This is developed further in \citet{reiners08} and also \citet{schmidt14}, however the latter disagree with
the previously-reported drop-off in activity for \ldwarf s, arguing that dust affects the measurements of activity of
these and show activity in those down to about L6.

In \citet{mohanty02} the relation between activity and rotational velocity is explored, however both in this and in
\citet{mohanty03} the authors state that the correlation between rotation and activity are not as strongly correlated in
fully convective stars from M3 on and suggest a ``turbulent dynamo'' responsible for the magnetic field generation.

It is clear that the measurement of rotational velocity is of importance in a full understanding of these processes. In
addition an understanding of periodicity is important for the search and detection of exoplanets. Recent papers have
expressed quite divergent views regarding whether reported planets have been validly detected by their period (see
\citealt{barnes13,robertson14,robertson14a,tuomi13aug,robertson15}). 

\section{The importance of {\ha}}
\protect\label{section:intohalpha}

The {\ha} line is a powerful diagnostic tool in many areas of astronomy. For example, {\ha} luminosity is one of the key
ways of estimating the star formation rates in galaxies (see for example \citet{rosagonzalez02}).

In terms of the study of stars, since the core of {\ha} forms above the photosphere, as shown in \citep{vernazza81}, it
is an important line for studying activity and variability in stellar chromospheres \citep{hall08}. Chromospheric
features such as plage regions and jets are seen well in {\ha} (see for example \citet{kneer10} and \citet{kuridze11}). 
The behaviour of the {\ha} line in {\rdwarf} stars is seen as a key diagnostic of activity, as the previously-cited
papers (\citealt{mohanty02, mohanty03, reiners08, schmidt14}) in connection with the relationship between rotation and
activity all show.

The behaviour of the {\ha} line is a potentially important diagnostic because it is sensitive to magnetic activity and
is a strong line usually seen in emission in later {\rdwarf} stars. It is used as the measure of activity in, for
example \citet{mohanty03} and subsequent papers. In \citet{hatzes15} the {\ha} periodicity is explicitly studied in
relation to the much-discussed GL 581 and possible GL 581d.

This {\paperorthesis} investigates whether periodicity can be identified in the morphology of the {\ha} line obtainable
from high-resolution spectra such as those obtained from the \textit{Ultraviolet and Visual Echelle Spectrograph}
({\uves}) at the 8.2m Very Large Telescope (VLT, UT 2) and the \textit{High Accuracy Radial velocity Planet Searcher}
({\harps}) at the ESO La Silla 3.6m telescope. \citet{barnes14} noticed that while a strong transient emission in {\ha}
was observable during flares, the jitter was limited. However it must be said that in \citet{reiners09}, the authors
suggest that Ca, He and {\ha} lines, being chromospheric emissions, are strongly affected by activity and should be
ommitted when searching for radial velocities in active \rdwarf s. it is observed that that activity does not greatly
affect radial velocity jitter in \rdwarf s.

Conscious of these divergent views, this {\paperorthesis} attempts to study the merits of precision measurements using
chromospheric features and in particular {\ha} analysis.

\section{\prox}
\protect\label{section:intoprox}


This study focuses on {\prox} for a number of reasons. As the nearest star to the solar system at 1.3 parsecs it is
obviously of particular interest with extensive data sets available. The {\ha} line is always in emission with a
characteristic shape. Thus is it a good place to test the general methodology applicable to late \rdwarf s.

Previous studies of the rotation period of {\prox} have given a value initially suggested as 31.5 $\pm$ 1.5 days in
\citet{guinan96}, a value of 41.3 days in \citet{benedict93} and as between 82 and 84 days in
\citealt{benedict92,benedict98}, confirmed as being not less than 50 days in \citet{kurster99} and as 82.5 days of
\citet{kiraga07}. Latterly it has been given as 116.6 days in \citet[Table 3]{suarezmascareno15}.  {\prox} has quite
frequent flares and the presence of these flares is useful for determination of whether and to what extent they disturb
methods for the calculation of the rotation period. The previous studies of rotation periods have found variations in
the overall intensity as in \citet{benedict98} and \citet{kiraga07}. In \citet{cincunegui07} is reported a a 447-day
activity cycle.In this paper {\Firstp} make use of archival high resolution spectroscopic observations to study periodic
changes in the detailed morphology of the {\ha} line and evolve additional techniques.

In chapter \ref{chapter:photometry} {\Firstp} discuss photometric measurements of periodicity from \prox, as these prove a
useful benchmark for spectroscopic measurements. In section \ref{chapter:proxima} {\Firstp} introduce the measurement
methods which {\Firstp} study in this paper and subsequently use to investigate periodicity from sets of spectra. In
section \ref{chapter:modelling} {\Firstp} go on to generate possible models of {\prox} and attempt to apply the
measurement methods to simulated spectra to evaluate their performance, obtain an error estimate and how this varies
with poorer signal to noise ratio. Sections \ref{chapter:discussion} and \ref{chapter:conclusions} report {\Firstposs}
discussion and conclusions.


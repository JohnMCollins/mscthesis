\chapter{Modelling of {\prox} spectra}
\protect\label{chapter:modelling}
\lhead{Chapter 4. \emph{Modelling of {\prox} spectra}}

In order to develop and refine the methods for evaluation of the periodicity of the sub-peaks in the {\prox} spectra,
{\Firstp} used a version of the ``Doppler Tomography of Stars'' (DoTS) modelling software
\citep{CCamerondotsa}. Although DoTS was written to recover surface imhomogeneities from time series spectra, here
{\Firstp} use the forward modelling routines to generate synthetic spectra, with some modifications. Specifically,
{\Firstp} construct a 3D model of the star, covered in a finite number of pixels. The intensity of each pixel can vary
from a photospheric value to a value appropriate for plage. In order to obtain the appropriate photospheric intensity
for each pixel at a given rotation phase of the 3D stellar model, {\Firstp} used the 4-parameter limb darkening law
introduced by Claret from Phoenix model atmospheres \citep{claret00a} for an effective temperature of 3000K. The plage
intensities were calculated according to \citet[Section 4.1]{unruh99}, who identified the centre to limb variability
from plage regions relative to the photospheric (quiet) intensity for the Sun. Since no such observations exist for
other stars, {\Firstp} adopted the same law with appropriate facular contrasts for {\ha} wavelengths (see \citet[figs 3
\& 4]{unruh99}).

Since {\Firstp} wish to simulate the {\ha} line profile, a local intensity profile is assumed for the photosphere and
the plage. For inactive photospheres of \rdwarf s with similar spectral type to {\prox}, {\ha} is not visible (e.g. see
{\ha} profile in \citet[fig. 6]{barnes14} for GJ1061 and GJ1062). Hence for the quiet photosphere, {\Firstp} assume a
flat continuum. For active stars, {\ha} possesses a characteristic emission profile with self-absorption, resulting in
a double-peaked profile. Since the \textit{vsini} is probably less than 0.1 km/s for \prox, {\Firstp} based the local
line profile shape for {\ha} on the observed {\prox} line profile since it is unlikely to show rotational
broadening. This profile was tuned to resemble the average {\ha} profile shown in the {\uves} data analysed in
\citet{fuhrmeister11}, but symmetric about the central wavelength. Specifically, {\Firstp} used a Gaussian profile to
generate the emission peak and a second Gaussian with narrow width to represent the central self-absorption as
illustrated in Fig. \ref{fig:integregions}.

\begin{figure}[!htbp]
\begin{center}
\includegraphics[scale=0.25]{Figures/integregions.png} \\
\end{center}
\caption{Example generated model spectrum of \prox, also illustrating the methods for computing the periodicity of
  spectra.  The centre of the \ha{} line is set at 6563{\AA} for convenience rather than 6562.8\AA{}.  The green lines
  (from 6561.75{\AA} to 6564.25\AA) show the limits used for calculation of the equivalent width. The blue and red
  shaded areas (6561.95{\AA} to 6562.75{\AA} and 6563.44{\AA} to 6564.24{\AA} respectively, each 0.8{\AA} wide) show the
  regions for calculation of the peak ratio.}
\protect\label{fig:integregions}
\end{figure}
% Done with 10 plage 80 period 75 deg first spectrum

With {\Firstposs} two-temperature model, in subsequent simulations below, {\Firstp} assign either photospheric intensity
or a plage intensity to each pixel. For a pixel containing plage, {\Firstp} thus scale the synthetic {\ha} profile and
for the photosphere with no visible profile (as note above), {\Firstp} use the continuum level. The line profile is
shifted appropriately for the Doppler shift of each pixel in {\Firstposs} model. The model enables {\Firstobj} to place
circular spots of specified radii anywhere on the star. For each viewing angle (or equivalently observation phase),
{\Firstp} calculate the appropriate intensity profiles of all visible pixels (according to position on the line and
centre-to-limb variation) and sum them to obtain our simulated line profile.

A model star with plage regions that rotate into and out of view can thus potentially exhibit variability in the line
shape since the pixels on different parts of the star possess different Doppler velocities. For stars such as \prox,
which probably possess a \textit{vsini} much less than the instrumental resolution, any distortions in the the line
profile due to spots rotating into and out of view may be insignificant or very small. A plage region that rotates into
view may nevertheless have a significant effect on the the equivalent width of the simulated line since {\Firstposs}
local intensity profile for {\ha} possesses a normalised peak intensity of N times the continuum. For stars with
rotational velocity much greater than the instrument resolution, line asymmetries are likely to be much more reliable.

\section{Plage distribution and results}
During the course of experimentation with models, {\Firstp} tried a selection of plage distributions, ranging from a
single large spot on one face to a heavily covered face as illustrated in the sixth image in
\citet[Fig. 1]{barnes11}. However {\Firstp} found that the variation in equivalent widths from a low spot coverage was
bore no possible resemblance to that from observational data, in that just exhibited two extremes of Equivalent Widths
and no intermediate values. On the other hand a coverage of more than about 30\% provided very limited swings in the
Equivalent Width compared those observed from {\harps} and {\uves}. After some experimentation, {\Firstp} settled for a
randomly distributed plage which filled up to 2.5\% of the surface, towards the high end of the coverage of up to 2.7\%
reported in \citet{guttenbrunner14} in relation to the Sun.
% The second was the same combined with a highly fairly asymmetric
% additional 1\% of coverage in a wedge between 0{\degree} and 37{\degree}. To avoid a concentration at the poles the
%likelihood of selecting a latitude point was made to be approximately proportional to the cosine of the latitude.

{\FirstP} generated the models with the observation dates from {\harps}. {\FirstP} varied possible rotation periods
between 15 and 90 days in steps of 5 days and inclinations between 10{\degree} and 90{\degree} in steps of 5{\degree} to
observe the various effects. Only a limited selection, usually multiples of 10 days and 10{\degree} are shown in this
{\paperorthesis} to conserve space.

\IfThesis{In Fig. \ref{fig:extremew} some extremes are shown.

\begin{figure}[!htbp]
\begin{center}
\includegraphics[scale=0.25]{Figures/extremes.png} \\
\end{center}
\caption{This illustrates the effects of extremes of symmetry of plage data on the variations in equivalent width. In
  each case a rotation period of 80 days and a stellar inclination of 90{\degree} is used. In the case on the left 100 plage
  points are distributed randomly over the face of the star whilst in the case on the right the same number of plage
  points are distributed over a 30{\degree} longitude band on one face of the star. The times from the {\harps} data are
  used in the model and all the generated spectra superimposed.} 
\protect\label{fig:extremew}
\end{figure}
}

{\FirstP} present in table \ref{table:modelcomp} a selection of typical results showing means and standard deviations
for the Equivalent Width method (EW)) with the two plage distributions for various periods and with 30{\degree},
60{\degree} and 90{\degree} inclinations and for 70, 80 and 90-day periods with 10{\degree} to 90{\degree} inclinations.
Note that the PR results are not displayed as variations were insufficient to be displayed in less than 6
figures, the Doppler variations are just not substantial enough.

\begin{table}[!htbp]
\centering
\scalebox{0.50}{
\begin{tabular}{|l|c|l|l|l|}
\hline
Plage Dist & Period & \multicolumn{1}{|c|}{30\degree}&\multicolumn{1}{|c|}{60\degree}&\multicolumn{1}{|c|}{90\degree}\\\hline
\multirow{8}{*}{Random to 2.5\%} & 20 & 2.150 $ \pm $ 0.572 & 2.746 $ \pm $ 0.939 & 2.735 $ \pm $ 0.891  \\
& 30 &  2.725 $ \pm $ 0.495 & 3.870 $ \pm $ 0.895 &  4.004 $ \pm $ 0.942 \\
& 40 &  1.917 $ \pm $ 0.541 & 2.501 $ \pm $ 0.927 &  2.614 $ \pm $ 0.967 \\
& 50 &  1.604 $ \pm $ 0.483 & 1.996 $ \pm $ 0.805 &  2.018 $ \pm $ 0.876 \\
& 60 &  1.626 $ \pm $ 0.573 & 2.057 $ \pm $ 0.951 &  2.099 $ \pm $ 1.006 \\
& 70 &  1.967 $ \pm $ 0.445 & 2.334 $ \pm $ 0.803 &  2.340 $ \pm $ 0.821 \\
& 80 &  1.637 $ \pm $ 0.495 & 2.161 $ \pm $ 0.805 &  2.309 $ \pm $ 0.828 \\
& 90 &  2.475 $ \pm $ 0.548 & 3.365 $ \pm $ 0.894 &  3.410 $ \pm $ 0.901 \\\hline
\end{tabular}}

\vspace{5 mm}

\scalebox{0.50}{
\begin{tabular}{|l|c|l|l|l|}
\hline
Plage Dist & Incl\degree & \multicolumn{1}{|c|}{70 days}&\multicolumn{1}{|c|}{80 days}&\multicolumn{1}{|c|}{90 days}\\\hline
\multirow{9}{*}{Random to 2.5\%} & 10 & 1.693 $ \pm $ 0.140 & 1.484 $ \pm $ 0.182 & 1.816 $ \pm $ 0.196 \\
& 20 &  1.811 $ \pm $ 0.292 &  1.492 $ \pm $ 0.356 &  2.116 $ \pm $ 0.387 \\
& 30 &  1.967 $ \pm $ 0.445 &  1.637 $ \pm $ 0.495 &  2.475 $ \pm $ 0.548 \\
& 40 &  2.116 $ \pm $ 0.590 &  1.810 $ \pm $ 0.626 &  2.827 $ \pm $ 0.691 \\
& 50 &  2.237 $ \pm $ 0.711 &  1.994 $ \pm $ 0.732 &  3.139 $ \pm $ 0.815 \\
& 60 &  2.334 $ \pm $ 0.803 &  2.161 $ \pm $ 0.805 &  3.365 $ \pm $ 0.894 \\
& 70 &  2.392 $ \pm $ 0.860 &  2.277 $ \pm $ 0.848 &  3.477 $ \pm $ 0.930 \\
& 80 &  2.396 $ \pm $ 0.877 &  2.338 $ \pm $ 0.850 &  3.464 $ \pm $ 0.911 \\
& 90 &  2.340 $ \pm $ 0.821 &  2.309 $ \pm $ 0.828 &  3.410 $ \pm $ 0.901 \\\hline
\end{tabular}}
\caption{Simulated mean equivalent widths with associated standard deviations from simulations for the 2.7\%
  plage distributions and a set of rotation periods and inclinations. In the first table results are illustrated for
  various periods and for 30{\degree}, 60{\degree} and  90{\degree} inclinations. In the second table results are
  illustrated for various inclinations and 70, 80 and 90-day periods as these are close to the
  rotation period of \prox.}
\protect\label{table:modelcomp}
\end{table}

For each set of generated spectra for both plage distributions, rotation period (between 15 and 90 days in steps of 5
and inclination (between 10{\degree} and 90{\degree} in steps of 5\degree), {\Firstp} obtained a periodogram, viewing
periods between 10 days and 130 days in steps of 0.01 days, from the calculated equivalent widths and noted the RMS
error over all inclinations. In nearly all cases the error was rarely more than 0.02 days.
% The random
%plage to 2.5\% case performed rather worse than the asymmetric case and in three cases with the former and with a 65-day
%period failed to find the correct period. With an 80 day period and random plage the error was as high as 0.40 days in
%some cases with a RMS error of 0.32 days, still much less than 1\% error. The asymmetric case never returned an error
%higher than 0.03 days for any starting period or inclination.
% {\FirstP} also looked at the results for the peak ratio
% periodograms and found for this perfect case that the errors were very slightly better overall than for the equivalent
% width cases, with no period failing to be found. Given the very slight variation in the peak ratios delivered by the
% model, this seemed surprising.

\section{Adding in noise and flares}
\protect\label{section:addflares}

Despite the limitations of the simplistic model, it is clear that a good estimate of periodicity may be obtained from
the equivalent width method and also from the peak ratio method of calculation. However this is with a ``perfect'' model
and to compare with reality the performance of the methods in the presence of noise and also with simulated flare events
had to be considered.

As a first step in moving to something like actual observational data, {\Firstp} tried adding noise of a given signal to
noise ratio to the simulated spectra and observed the effect on the accuracy of the periodicity measurements for various
levels and inclinations. {\FirstP} tried adding Gaussian noise with SNRs from 40 down to 1 in steps of 0.1.
{\FirstP} tried this with all the combinations of inclinations and starting periods tried before.

It was noticeable that doing this only started to have any significant effect with SNR below 20. Below this level, two
things started to happen, increasingly as the SNR was reduced. Either the error in the recovered period increased,
although not by very much or the recovered period was either manifestly incorrect, giving a clear False Positive such as
returning a period of 50 days from a starting period of 80 days.

It was easy to discriminate between these two cases by setting a threshold of 5\% for the difference between the
recovered period and the starting period. If the difference exceeded this, then the period was regarded as incorrectly
recovered, otherwise it was regarded as correctly recovered but with the given error. However in all the cases the
difference was either substantially greater or substantially less than this. It was noticeable that in quite a number of
cases a period close to 116 days was returned, an example of such a case is shown in the right panel of
Fig. \ref{fig:noiseresults}.

{\FirstP} also examined the possible effect of flares. {\FirstP} simulated the effect of flares by taking the spectra
which were clipped as having excessive Equivalent Width in Section \ref{section:harpsper} and adding in the same
proportionate excess over the median Equivalent Width to the model as was found in the observed data. In
Fig. \ref{fig:noiseresults} is illustrated the effects of adding noise at various levels to the percentage of correct
recoveries with either the largest 4 flares listed in Section \ref{section:uvesflares}, or the ones clipped as having an
Equivalent Width greater than 3.8 (one standard deviation above the median).

In the left panel of Fig. \ref{fig:noiseresults} is shown the effect of varying SNR and adding in simulated flare data
on the rate of recovery of the correct period. The blue plot shows the effect of just adding noise, the green that of
adding in the four largest flares and noise and the red shows the effect of adding simulated flares from all the values
in the {\harps} data clipped having Equivalent Width over 3.8.

\Notnow{
\begin{table}[!htbp]
\centering
\scalebox{0.75}{
\begin{tabular}{|l|r|r|r|l|r|r|r|}
\hline
SNR & None & Peak 1 & Peaks 1-3 & SNR & None & Peak 1 & Peaks 1-3 \\\hline
25 & 99.50 & 99.00 & 98.40 & 10 & 97.20 & 96.20 & 93.80 \\\hline
20 & 99.50 & 98.80 & 97.90 & 9 & 95.30 & 93.40 & 90.80 \\\hline
19 & 99.30 & 99.10 & 98.30 & 8 & 94.60 & 91.00 & 87.90 \\\hline
18 & 99.30 & 98.80 & 98.10 & 7 & 93.80 & 88.90 & 83.00 \\\hline
17 & 98.60 & 98.30 & 98.10 & 6 & 90.50 & 84.30 & 78.70 \\\hline
16 & 99.50 & 99.00 & 97.90 & 5 & 88.90 & 82.50 & 78.50 \\\hline
15 & 99.70 & 98.60 & 97.90 & 4 & 86.90 & 79.80 & 72.70 \\\hline
14 & 99.00 & 98.80 & 97.20 & 3 & 82.40 & 72.00 & 63.70 \\\hline
13 & 99.00 & 97.40 & 96.50 & 2 & 80.60 & 66.40 & 60.40 \\\hline
12 & 98.40 & 98.10 & 96.40 & 1 & 73.70 & 58.00 & 52.20 \\\hline
1 & 97.10 & 95.30 & 94.10 & 0 & 69.70 & 54.80 & 45.50 \\\hline
\end{tabular}}
\caption{Percentage recovery with integer values of SNR for all inclinations, all starting periods and no flare, just
  largest flare and all the flares for {\harps} discussed in Fig. \ref{fig:proxhists}. Note that this table involves extensive use
  of randomly-generated values for plage distribution and noise and is not exactly reproducible.}
\protect\label{table:noiseresults}
\end{table}}

\begin{figure}[!htbp]
\begin{center}
\includegraphics[scale=0.25]{Figures/missedperiod.png} \\
\end{center}
\caption{This figure shows (left) the percent of correctly recovered periods up to 3\% for all periods and inclinations
  with noise added with SNR between 1 to 10 with no flare data added (blue), just the four largest flare data (green)
  and the flare data clipped in Section \ref{section:harpsper} as having Equivalent Width greater than 1 standard
  deviation from the median. One the right is shown an example of a periodogram from a model with an 80-day period,
  inclination of 80{\degree} no flare data and a SNR of 3. The correct period is missed in the highest peak signal
  (103.7 days) but appears as the second-highest peak (green) at 80.45 days. A further peak is returned as 116.5 days
  (red). Note that this figure heavily involves extensive use of randomly-generated noise and is not exactly
  reproducible.}
\protect\label{fig:noiseresults}
\end{figure}

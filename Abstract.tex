In this {\paperorthesis} the stellar rotation signal for {\prox} is investiaged. High-resolution spectra taken with
{\uves} and {\harps} of {\prox} over a 13-year period is used as well as photometric observations of {\prox} from
{\asas} and {\hst}.

The {\ha} equivalent width and {\ha} index are measured (and found to be very similar), together with other measurments,
  skewness and kurtosis and a method that investigates the symmetry of the line, the Peak Ratio, is introduced which
  appears to return better results than the other measurements.

The investigations return a most significant period of 82.6 days, confirming photometric results and ruling out a more
  recent result of 116.6 days which it is concluded is an artefact of the observation times.

It is also concluded that whilst spectroscopic {\ha} measurements can be used for period recovery, in the case of
  {\prox} the available photometric measurements are more reliable. By using 2D models of {\prox} to generate simulated
  {\ha}, it is found that reasonable distributions of plage and chromospheric features are able to reproduce the
  equivalent width variations in observed data and recover the rotation period. \examrevision{The period recovery is still effective
  after simulated noise and the effects of flares similar to those in the observed data the are added to the
  modelling results.}

However it is concluded only with the introduction of 3D models which allow for tracking of vertical motions can
simulations manage to generate the observed variety of line shapes measured, for example, by the peak ratio.

{\FirstP} investigate retrieval of the stellar rotation signal for {\prox}. {\FirstP} make use of high-resolution
spectra taken with {\uves} and {\harps} of {\prox} over a 13-year period as well as photometric observations of
{\prox} from {\asas} and {\hst}. {\FirstP} focus on the {\ha} Equivalent Width and {\ha} Index and introduce a method
that investigates the symmetry of the line, the Peak Ratio, which appears to return better results than the other
measurements.

{\FirstPoss} investigations return a most significant period of 82.6 days, confirming photometric
results and rule out a more recent result of 116.6 days. {\FirstP} conclude that whilst spectroscopic measurements are
useful, they cannot reliably be used in isolation, at least as far as {\prox} is concerned.

{\FirstP} also consider
models of {\prox} to generate simulated spectra with various distributions of plage and chromospheric features and are
able to reproduce the Equivalent Width variations observed in the {\harps} data and recover the rotation period of the
spectra, assessing the stability in the face of emulated noise and flares. However it was hard to reproduce the line
asymmetry measured by the Peak Ratio measure with the models.

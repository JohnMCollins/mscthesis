\chapter{Full periodogram results} % Main appendix title
\protect\label{chapter:pgramdetail}
\lhead{Appendix \ref{chapter:pgramdetail}. \emph{Periodogram details}}

In this appendix the complete results for the periodicity studies with various treatments of the data and Python
routines are presented.

\section{Equivalent Widths}
\protect\label{section:appewtab}

The following tables, Table \ref{table:origewtaball} and \ref{table:fullewtaball}, present all the periodogram
calculations with the three Python routines employed, with the Original Set of {\harps} data to January 2014 and the
Full Set to March 2016 respectively. The Equivalent Widths are calculated with the Original Set to highlight that the 116-day
period found in \citet{suarezmascareno15} is also returned with those, but not with the Full Set to January 2014.

\begin{table}[!htbp]
\centering
\scalebox{0.75}{
\begin{tabular}{|l|l|l|r|r|r|r|r|}
\hline
\textbf{Treatment}&\textbf{Points}&\textbf{Routine}&\textbf{Peak 1}&\textbf{Peak 2}&\textbf{Peak 3}&\textbf{Peak 4}&\textbf{Peak 5}\\\hline
None & \multirow{3}{*}{260} & \scipy & 116.2 & \textit{41.4} & 52.1 & 46.2 & 48.9 \\
 && \astroml & 44.9 & 57.6 & 59.4 & 104.2 & 84.4 \\
 && \gatspy & 49.1 & \textit{41.4} & 49.9 & 57.0 & 116.3 \\\hline
Clipped 1$\sigma$ & \multirow{3}{*}{237} & \scipy & 118.4 & 57.7 & 106.8 & 105.3 & 44.9 \\
 && \astroml & 45.0 & 51.2 & 57.7 & 47.4 & 84.5 \\
 && \gatspy & 44.9 & 118.4 & 57.8 & 107.2 & 46.4 \\\hline
Binned & \multirow{3}{*}{57} & \scipy & \textit{41.4} & 53.2 & 110.5 & 62.6 & 116.1 \\
1 day && \astroml & 115.5 & 87.7 & 71.1 & 59.4 & 40.0 \\
 && \gatspy & \textit{41.4} & 53.2 & 110.5 & 62.6 & 49.9 \\\hline
Binned & \multirow{3}{*}{89} & \scipy & \textit{41.4} & 110.4 & 53.1 & 49.0 & 49.9 \\
0.5 day && \astroml & 45.0 & 115.0 & 59.4 & 110.3 & \textbf{82.6} \\
 && \gatspy & \textit{41.4} & 49.0 & 57.0 & 59.3 & 49.9 \\\hline
Clipped 1$\sigma$ & \multirow{3}{*}{55} & \scipy & 45.0 & 59.5 & 71.4 & 89.0 & 109.7 \\
binned && \astroml & 59.5 & 45.0 & 71.3 & 49.9 & 51.2 \\
1 day && \gatspy & 45.0 & 59.5 & 71.4 & 89.0 & 107.7 \\\hline
Clipped 1$\sigma$ & \multirow{3}{*}{83} & \scipy & 45.0 & 59.5 & 51.2 & 71.3 & 117.3 \\
binned && \astroml & 45.0 & 51.2 & 40.0 & 59.5 & 57.9 \\
0,5 day && \gatspy & 45.0 & 59.5 & 51.3 & 71.5 & 117.3 \\\hline
Residuals & \multirow{3}{*}{260} & \scipy & 116.2 & \textit{41.4} & 48.9 & 52.1 & 53.0 \\
 && \astroml & 45.0 & 59.4 & 57.6 & 84.3 & 104.2 \\
 && \gatspy & 49.1 & \textit{41.4} & 49.9 & 57.0 & 116.3 \\\hline
\end{tabular}}
\caption{This table shows the 5 highest peaks from the periodograms for Equivalent Widths with various treatments of the
  Original Set of data to January 2014. Highlighted in bold are periods close to 82.6 days and in italics periods close
  to 41.3 days.}
\protect\label{table:origewtaball}
\end{table}

\begin{table}[!htbp]
\centering
\scalebox{0.75}{
\begin{tabular}{|l|l|l|r|r|r|r|r|}
\hline
\textbf{Treatment}&\textbf{Points}&\textbf{Routine}&\textbf{Peak 1}&\textbf{Peak 2}&\textbf{Peak 3}&\textbf{Peak 4}&\textbf{Peak 5}\\\hline
None & \multirow{3}{*}{316} & \scipy & 49.0 & \textit{41.4} & 62.6 & 56.8 & 53.2 \\
 && \astroml & 114.6 & 123.2 & 108.0 & \textbf{83.1} & 91.8 \\
 && \gatspy & 59.2 & 49.0 & 62.6 & 56.9 & \textit{41.4} \\\hline
Clipped 1$\sigma$ & \multirow{3}{*}{287} & \scipy & 122.2 & 108.0 & \textbf{83.1} & 90.2 & 99.1 \\
 && \astroml & 123.1 & 114.9 & 107.8 & \textbf{83.1} & 91.8 \\
 && \gatspy & 122.5 & 108.0 & 59.1 & 48.1 & 50.1 \\\hline
Clipped & \multirow{3}{*}{277} & \scipy & 108.0 & 122.0 & \textbf{83.0} & 90.0 & 120.1 \\
To 3.8 EW && \astroml & 123.3 & 114.7 & 107.9 & \textbf{83.2} & 87.0 \\
 && \gatspy & 108.0 & 122.2 & 58.1 & \textbf{83.1} & 89.9 \\\hline
Binned & \multirow{3}{*}{93} & \scipy & \textit{41.4} & 49.0 & 62.6 & 53.3 & 59.2 \\
1 day && \astroml & 123.0 & 114.5 & 91.7 & 101.1 & 107.8 \\
 && \gatspy & \textit{41.4} & 59.2 & 49.0 & 62.6 & 53.3 \\\hline
Binned & \multirow{3}{*}{143} & \scipy & 49.0 & 59.2 & \textit{41.4} & 62.6 & 56.8 \\
0.5 day && \astroml & 114.5 & 123.2 & 91.8 & \textbf{83.1} & 108.0 \\
 && \gatspy & 59.2 & 49.0 & \textit{41.4} & 62.6 & 56.8 \\\hline
Clipped 1$\sigma$ & \multirow{3}{*}{89} & \scipy & 114.9 & 123.3 & 108.0 & 87.0 & \textbf{83.2} \\
binned && \astroml & 114.8 & 123.2 & 91.7 & 107.7 & 87.0 \\
1 day && \gatspy & 114.9 & 123.3 & 108.0 & 87.0 & \textbf{83.2} \\\hline
Clipped & \multirow{3}{*}{85} & \scipy & 114.8 & 108.0 & 123.4 & 86.9 & \textbf{83.1} \\
To 3.8 EW && \astroml & 114.6 & 123.2 & 107.7 & 91.7 & 86.9 \\
binned 1 day && \gatspy & 114.8 & 108.0 & 123.4 & 86.9 & \textbf{83.1} \\\hline
Clipped 1$\sigma$ & \multirow{3}{*}{129} & \scipy & 114.8 & 123.4 & 108.1 & \textbf{83.2} & 87.0 \\
binned && \astroml & 114.8 & 123.1 & 91.9 & 107.7 & \textbf{83.2} \\
0.5 day && \gatspy & 114.9 & 123.4 & 108.0 & \textbf{83.2} & 54.9 \\\hline
Clipped & \multirow{3}{*}{122} & \scipy & 114.7 & 108.1 & 123.4 & 86.9 & \textbf{83.2} \\
To 3.8 EW && \astroml & 114.7 & 123.3 & 107.9 & 92.0 & 87.0 \\
binned 0.5 day && \gatspy & 114.8 & 108.0 & 123.4 & 86.9 & \textbf{83.2} \\\hline
Residuals & \multirow{3}{*}{316} & \scipy & 49.0 & \textit{41.4} & 62.6 & 56.8 & 53.3 \\
 && \astroml & 114.6 & 123.2 & 108.0 & \textbf{83.1} & 91.8 \\
 && \gatspy & 59.2 & 49.0 & 62.6 & 56.9 & \textit{41.4} \\\hline
\end{tabular}}
\caption{This table shows the 5 highest peaks from the periodograms for Equivalent Widths with various treatments of the
  Full Set of data to March 2916. Highlighted in bold are periods close to 82.6 days and in italics periods close
  to 41.3 days.}
\protect\label{table:fullewtaball}
\end{table}

In these tables where spectra are clipped as having Equivalent Width over one standard deviation above the median, this
is to the values given in Table \ref{table:ewtabfirst}, i.e. 3.8 for the Original Set to 2014 and 4.2 for the Full Set
to 2016.

The residual Equivalent Widths are calculated by dividing by the mean of the 5 spectra with the smallest {\ha}
Equivalent Widths. These spectra were ones timed at 5 April 2011 UTC 03:26:33 (the lowest), 16 March 2006 UTC 06:37:59,
14 March 2007 UTC 07:28:29, 8 April 2011 UTC 06:28:17 and 22 April 2011 UTC 05:07:46.

\section{{\ha} Index measurement}
\protect\label{section:apphaitab}

In Table \ref{table:bothhaitable} is shown the results for all three Python routines used evaluating the {\ha} Index.
These are calculated with the Original Set of data to 2014 as well as the Full Set to 2016 to highlight that the 116-day
period found in \citet{suarezmascareno15} with the Original Set, but not with the Full Set to January 2014.

\begin{table}[!htbp]
\centering
\scalebox{0.75}{
\begin{tabular}{|l|l|r|r|r|r|r|}
\hline
\textbf{Data}&\textbf{Routine}&\textbf{Peak 1}&\textbf{Peak 2}&\textbf{Peak 3}&\textbf{Peak 4}&\textbf{Peak 5}\\\hline
Set to 2014 & \scipy & 116.2 & 41.4 & 52.1 & 48.9 & 53.0 \\
 & \astroml & 45.0 & 51.2 & 57.6 & 84.5 & 116.6 \\
 & \gatspy & 49.1 & 41.4 & 49.9 & 57.0 & 116.3 \\\hline
Full Set & \scipy & 49.0 & 41.4 & 62.6 & 56.8 & 53.3 \\
 & \astroml & 123.2 & 114.7 & 107.9 & 83.1 & 87.0 \\
 & \gatspy & 59.2 & 49.0 & 62.6 & 56.9 & 41.4 \\\hline
\end{tabular}}
\caption{This table shows the 5 highest peaks from the periodograms for the Original and Full Sets of Data.}
\protect\label{table:bothhaitable}
\end{table}

\section{{\ha} Peak Ratio measurments}
\protect\label{section:appprtab}

The following tables, Table \ref{table:origprtaball} and \ref{table:fullprtaball}, present all the periodogram
calculations with the three Python routines employed, with the Original Set of {\harps} data to January 2014 and the
Full Set to March 2016 respectively. The Peak Ratios are calculated with the Original Set to highlight that the 116-day
period found in \citet{suarezmascareno15} for the {\ha} Index measurement (and also with the Equivalent Widths, see
Section \ref{section:appewtab} is not returned with this measurement.

\begin{table}[!htbp]
\centering
\scalebox{0.75}{
\begin{tabular}{|l|l|l|r|r|r|r|r|}
\hline
\textbf{Treatment}&\textbf{Points}&\textbf{Routine}&\textbf{Peak 1}&\textbf{Peak 2}&\textbf{Peak 3}&\textbf{Peak 4}&\textbf{Peak 5}\\\hline
None & \multirow{3}{*}{260} & \scipy & \textit{41.9} & 92.4 & \textbf{81.1} & 49.7 & 44.1 \\
 && \astroml & 80.0 & 103.9 & 92.5 & 49.2 & 45.5 \\
 && \gatspy & \textbf{81.0} & \textit{41.9} & 92.4 & 103.9 & 44.1 \\\hline
Clipped 1$\sigma$ & \multirow{3}{*}{237} & \scipy & \textit{41.9} & \textbf{81.3} & 49.8 & 92.1 & 44.1 \\
 && \astroml & 49.9 & \textit{41.9} & 91.9 & 101.5 & 58.1 \\
 && \gatspy & \textbf{81.2} & \textit{41.9} & 49.8 & 92.2 & 44.1 \\\hline
Binned & \multirow{3}{*}{57} & \scipy & \textbf{82.2} & 61.0 & 80.9 & 49.6 & \textit{40.6} \\
1 day && \astroml & 71.1 & 115.2 & 80.0 & 91.8 & 123.2 \\
 && \gatspy & \textbf{82.3} & 61.0 & 49.4 & \textbf{81.0} & \textit{40.6} \\\hline
Binned & \multirow{3}{*}{89} & \scipy & 122.2 & 120.3 & \textbf{82.9} & 79.7 & 108.4 \\
0.5 day && \astroml & 80.5 & 91.9 & 115.0 & 104.0 & 109.6 \\
 && \gatspy & 122.1 & \textbf{83.0} & 79.8 & 108.4 & 78.1 \\\hline
Clipped 1$\sigma$ & \multirow{3}{*}{55} & \scipy & \textbf{82.2} & 61.1 & \textit{40.5} & \textbf{81.0} & 115.8 \\
binned && \astroml & 108.0 & 71.2 & 80.0 & 57.1 & 65.5 \\
1 day && \gatspy & \textbf{82.2} & 61.1 & \textit{40.5} & \textbf{81.0} & 49.3 \\\hline
Clipped 1$\sigma$ & \multirow{3}{*}{83} & \scipy & 122.2 & 120.2 & \textbf{82.6} & 108.3 & 78.2 \\
binned && \astroml & 108.8 & 57.8 & 47.3 & 103.5 & 54.4 \\
0,5 day && \gatspy & 122.2 & 120.5 & \textbf{82.7} & 108.3 & 79.7 \\\hline
Residuals & \multirow{3}{*}{260} & \scipy & \textit{41.9} & \textit{41.4} & 49.7 & 92.4 & 44.2 \\
 && \astroml & 45.4 & 80.0 & \textbf{84.1} & 92.6 & 49.2 \\
 && \gatspy & \textit{41.9} & 45.5 & 49.8 & \textit{41.4} & 80.9 \\\hline
\end{tabular}}
\caption{This table shows the 5 highest peaks from the periodograms for Peak Ratios with various treatments of the
  Original Set of data to January 2014. Highlighted in bold are periods close to 82.6 days and in italics periods close
  to 41.3 days.}
\protect\label{table:origprtaball}
\end{table}

\begin{table}[!htbp]
\centering
\scalebox{0.75}{
\begin{tabular}{|l|l|l|r|r|r|r|r|}
\hline
\textbf{Treatment}&\textbf{Points}&\textbf{Routine}&\textbf{Peak 1}&\textbf{Peak 2}&\textbf{Peak 3}&\textbf{Peak 4}&\textbf{Peak 5}\\\hline
None & \multirow{3}{*}{316} & \scipy & 121.7 & 123.3 & \textbf{83.0} & 114.6 & 90.1 \\
 && \astroml & 60.6 & 52.1 & 49.1 & 56.9 & 57.9 \\
 && \gatspy & 121.8 & 49.9 & 114.7 & \textbf{83.0} & \textit{42.0} \\\hline
Clipped 1$\sigma$ & \multirow{3}{*}{287} & \scipy & 122.3 & 91.8 & \textbf{83.0} & 115.0 & \textit{42.0} \\
 && \astroml & 60.8 & 73.0 & 102.0 & 79.5 & 56.8 \\
 && \gatspy & 122.5 & 58.1 & 47.5 & \textbf{83.0} & 91.7 \\\hline
Clipped & \multirow{3}{*}{277} & \scipy & 122.2 & 114.9 & 91.8 & \textbf{82.9} & 49.8 \\
To 3.8 EW && \astroml & 101.8 & 73.1 & 114.3 & 79.3 & 91.2 \\
 && \gatspy & 122.5 & 49.9 & 58.0 & 115.0 & 91.8 \\\hline
Binned & \multirow{3}{*}{93} & \scipy & \textbf{82.2} & \textit{40.6} & 51.1 & 63.6 & 66.8 \\
1 day && \astroml & 91.6 & 60.9 & 87.0 & 79.3 & 122.0 \\
 && \gatspy & \textbf{82.2} & \textit{40.6} & 51.1 & 63.6 & 48.0 \\\hline
Binned & \multirow{3}{*}{143} & \scipy & 120.4 & 121.7 & \textbf{82.3} & 107.6 & 117.5 \\
0.5 day && \astroml & 54.4 & 60.6 & 64.1 & 73.0 & 67.3 \\
 && \gatspy & 120.7 & \textbf{82.3} & 107.5 & 117.4 & 80.0 \\\hline
Clipped 1$\sigma$ & \multirow{3}{*}{89} & \scipy & \textit{40.5} & 80.9 & 48.1 & 116.7 & 51.1 \\
binned && \astroml & 114.8 & 91.4 & 122.1 & 122.7 & 107.4 \\
1 day && \gatspy & \textit{40.5} & 80.9 & 48.1 & 119.9 & 116.7 \\\hline
Clipped & \multirow{3}{*}{85} & \scipy & \textit{40.5} & 80.9 & 116.4 & \textbf{81.8} & 48.0 \\
To 3.8 EW && \astroml & 114.3 & 59.8 & 54.5 & 121.9 & 107.1 \\
binned 1 day && \gatspy & \textit{40.5} & 80.9 & 116.5 & \textbf{81.9} & 48.0 \\\hline
Clipped 1$\sigma$ & \multirow{3}{*}{129} & \scipy & 121.8 & 120.4 & \textbf{82.4} & 107.7 & 117.3 \\
binned && \astroml & 60.8 & 73.0 & 69.0 & 91.2 & 54.4 \\
0.5 day && \gatspy & 120.8 & 121.3 & \textbf{82.4} & 107.6 & 117.3 \\\hline
Clipped & \multirow{3}{*}{122} & \scipy & 121.8 & 120.4 & \textbf{82.4} & 107.7 & 117.2 \\
To 3.8 EW && \astroml & 91.2 & 67.4 & 73.0 & 101.8 & 69.0 \\
binned 0.5 day && \gatspy & 120.7 & 121.3 & \textbf{82.4} & 107.6 & 117.2 \\\hline
Residuals & \multirow{3}{*}{316} & \scipy & 121.6 & 123.7 & 49.8 & 90.0 & \textbf{83.0} \\
 && \astroml & 60.6 & 73.1 & 79.6 & 65.0 & 57.0 \\
 && \gatspy & 121.7 & 49.9 & 123.7 & 58.0 & 47.5 \\\hline
\end{tabular}}
\caption{This table shows the 5 highest peaks from the periodograms for Peak Ratios with various treatments of the
  Full Set of data to March 2916. Highlighted in bold are periods close to 82.6 days and in italics periods close
  to 41.3 days.}
\protect\label{table:fullprtaball}
\end{table}

In these tables the clipping, binning and calculation of residuals is exactly as described above for Equivalent Widths
in Section \ref{section:appewtab}.

\section{Skewness and Kurtosis measurements}
\protect\label{section:appkstab}

The following tables, Table \ref{table:origskewtaball} and \ref{table:origkurttaball} present the periodogram
calculations with the three Python routines employed, with the Original Set of {\harps} data to January 2014. Following
these the Tables \ref{table:fullskewtaball} and \ref{table:fullkurttaball} present the periodogram calculations for the
Full Set of data to March 2016.

As with the Equivalent Widths in Section \ref{section:appewtab}, the calculation is performed for the Original Set of data as
the 116-day period found in \citet{suarezmascareno15} again appears, even in this case after clipping, binning or both
is attempted.

\begin{table}[!htbp]
\centering
\scalebox{0.75}{
\begin{tabular}{|l|l|r|r|r|r|r|}
\hline
\textbf{Treatment}&\textbf{Routine}&\textbf{Peak 1}&\textbf{Peak 2}&\textbf{Peak 3}&\textbf{Peak 4}&\textbf{Peak 5}\\\hline
None & \scipy & 122.8 & 98.3 & 87.6 & 117.2 & 106.0 \\
 & \astroml & 87.7 & 116.8 & 105.5 & 122.4 & 79.7 \\
 & \gatspy & 87.7 & 116.8 & 105.5 & 122.4 & 79.7 \\\hline
Clipped 1$\sigma$ & \scipy & 98.2 & 122.9 & 65.3 & 71.3 & 59.7 \\
 & \astroml & 87.5 & 65.5 & 115.8 & 79.8 & 98.2 \\
 & \gatspy & 87.5 & 65.5 & 115.8 & 79.8 & 98.2 \\\hline
Binned & \scipy & 105.7 & 117.7 & 87.8 & 103.3 & 79.6 \\
1 day & \astroml & 105.7 & 117.7 & 87.8 & 103.3 & 48.8 \\
 & \gatspy & 105.7 & 117.7 & 87.8 & 103.3 & 48.8 \\\hline
Binned & \scipy & 105.8 & 87.7 & 117.3 & 79.6 & \textbf{81.2} \\
0.5 day & \astroml & 105.7 & 87.7 & 117.4 & 79.6 & \textbf{81.2} \\
 & \gatspy & 105.7 & 87.7 & 117.4 & 79.6 & \textbf{81.2} \\\hline
Clipped 1$\sigma$ & \scipy & 59.6 & 87.7 & 117.1 & 65.5 & 105.6 \\
binned & \astroml & 59.6 & 87.7 & 117.0 & 65.5 & 105.6 \\
1 day & \gatspy & 59.6 & 87.7 & 117.0 & 65.5 & 105.6 \\\hline
Clipped 1$\sigma$ & \scipy & 87.6 & 116.5 & 65.5 & 59.7 & 105.5 \\
binned & \astroml & 87.6 & 65.5 & 116.5 & 59.7 & 105.5 \\
0,5 day & \gatspy & 87.6 & 65.5 & 116.5 & 59.7 & 105.5 \\\hline
Residuals & \scipy & 71.8 & 73.8 & 57.7 & 93.0 & 98.4 \\
 & \astroml & 93.1 & \textbf{83.7} & 74.1 & 52.2 & 71.9 \\
 & \gatspy & 93.1 & \textbf{83.7} & 74.1 & 52.2 & 71.9 \\\hline
\end{tabular}}
\caption{This table shows the 5 highest peaks from the periodograms for Skewness with various treatments of the Original
  Set of data to January 2014. Highlighted in bold are periods close to 82.6 days and in italics periods close
  to 41.3 days.}
\protect\label{table:origskewtaball}
\end{table}

\begin{table}[!htbp]
\centering
\scalebox{0.75}{
\begin{tabular}{|l|l|r|r|r|r|r|}
\hline
\textbf{Treatment}&\textbf{Routine}&\textbf{Peak 1}&\textbf{Peak 2}&\textbf{Peak 3}&\textbf{Peak 4}&\textbf{Peak 5}\\\hline
None & \scipy & 117.2 & 122.7 & 106.4 & 87.2 & 103.6 \\
 & \astroml & 116.7 & 80.9 & 87.3 & 105.2 & 103.7 \\
 & \gatspy & 116.7 & 80.9 & 87.3 & 105.2 & 103.7 \\\hline
Clipped 1$\sigma$ & \scipy & 117.2 & 87.3 & 59.6 & 65.4 & 122.5 \\
 & \astroml & 116.8 & 80.9 & 87.5 & 65.5 & 59.8 \\
 & \gatspy & 116.8 & 80.9 & 87.5 & 65.5 & 59.8 \\\hline
Binned & \scipy & 116.5 & 105.3 & 88.8 & 103.7 & 87.6 \\
1 day & \astroml & 116.4 & 105.3 & 103.8 & 88.8 & 87.6 \\
 & \gatspy & 116.4 & 105.3 & 103.8 & 88.8 & 87.6 \\\hline
Binned & \scipy & 116.6 & 87.5 & 88.4 & 105.4 & 80.5 \\
0.5 day & \astroml & 116.6 & 87.5 & 105.3 & 80.8 & 88.4 \\
 & \gatspy & 116.6 & 87.5 & 105.3 & 80.8 & 88.4 \\\hline
Clipped 1$\sigma$ & \scipy & 59.6 & 117.9 & 45.0 & 80.9 & 87.8 \\
binned & \astroml & 59.6 & 118.1 & \textbf{81.0} & 105.7 & 45.0 \\
1 day & \gatspy & 59.6 & 118.1 & \textbf{81.0} & 105.7 & 45.0 \\\hline
Clipped 1$\sigma$ & \scipy & 117.2 & 59.6 & 87.7 & 65.6 & 80.8 \\
binned & \astroml & 117.5 & 59.6 & 87.7 & 65.6 & \textbf{81.0} \\
0,5 day & \gatspy & 117.5 & 59.6 & 87.7 & 65.6 & \textbf{81.0} \\\hline
Residuals & \scipy & 117.6 & 106.6 & 103.5 & 74.0 & 51.4 \\
 & \astroml & 117.4 & \textbf{81.1} & 106.3 & 57.3 & 103.5 \\
 & \gatspy & 117.4 & \textbf{81.1} & 106.3 & 57.3 & 103.5 \\\hline
\end{tabular}}
\caption{This table shows the 5 highest peaks from the periodograms for Kurtosis with various treatments of the Original
  Set of data to January 2014. Highlighted in bold are periods close to 82.6 days and in italics periods close
  to 41.3 days.}
\protect\label{table:origkurttaball}
\end{table}

\begin{table}[!htbp]
\centering
\scalebox{0.75}{
\begin{tabular}{|l|l|r|r|r|r|r|}
\hline
\textbf{Treatment}&\textbf{Routine}&\textbf{Peak 1}&\textbf{Peak 2}&\textbf{Peak 3}&\textbf{Peak 4}&\textbf{Peak 5}\\\hline
None & \scipy & 121.9 & \textbf{83.4} & 60.0 & 79.1 & 106.4 \\
 & \astroml & \textbf{83.4} & 122.2 & 60.0 & 63.4 & 77.7 \\
 & \gatspy & \textbf{83.4} & 122.2 & 60.0 & 63.4 & 77.7 \\\hline
Clipped 1$\sigma$ & \scipy & 121.9 & 99.1 & 59.9 & \textbf{83.3} & 65.1 \\
 & \astroml & 59.9 & 63.3 & 77.8 & \textbf{83.3} & 113.5 \\
 & \gatspy & 59.9 & 63.3 & 77.8 & \textbf{83.3} & 113.5 \\\hline
Clipped & \scipy & 121.9 & 65.1 & 99.0 & \textbf{83.3} & 59.9 \\
To 3.8 EW & \astroml & 60.0 & 86.1 & 113.7 & 77.7 & 122.2 \\
 & \gatspy & 60.0 & 86.1 & 113.7 & 77.7 & 122.2 \\\hline
Binned & \scipy & 106.8 & 79.2 & 85.8 & 121.5 & 64.8 \\
1 day & \astroml & 106.8 & 79.2 & 85.8 & 121.6 & 64.8 \\
 & \gatspy & 106.8 & 79.2 & 85.8 & 121.6 & 64.8 \\\hline
Binned & \scipy & 79.2 & 85.8 & 106.7 & 60.0 & \textbf{83.4} \\
0.5 day & \astroml & 85.8 & 106.7 & 79.2 & 60.0 & \textbf{83.4} \\
 & \gatspy & 85.8 & 106.7 & 79.2 & 60.0 & \textbf{83.4} \\\hline
Clipped & \scipy & 59.7 & 71.9 & 85.9 & \textit{41.4} & 77.6 \\
1$\sigma$ & \astroml & 59.7 & 71.9 & 85.9 & \textit{41.4} & 107.0 \\
binned 1 day & \gatspy & 59.7 & 71.9 & 85.9 & \textit{41.4} & 107.0 \\\hline
Clipped & \scipy & 59.9 & 86.0 & 113.5 & 72.0 & 77.7 \\
1$\sigma$ & \astroml & 59.8 & 86.0 & 113.4 & 72.0 & 77.7 \\
binned 0.5 day & \gatspy & 59.8 & 86.0 & 113.4 & 72.0 & 77.7 \\\hline
Clipped & \scipy & 59.7 & 71.9 & 85.9 & 106.8 & 113.7 \\
To 3.8 EW & \astroml & 59.7 & 71.9 & 85.9 & 106.9 & 113.6 \\
binned 1 day & \gatspy & 59.7 & 71.9 & 85.9 & 106.9 & 113.6 \\\hline
Clipped & \scipy & 86.0 & 113.6 & 59.9 & 77.7 & 72.0 \\
To 3.8 EW & \astroml & 86.0 & 113.5 & 59.9 & 77.7 & 72.0 \\
binned 0.5 day & \gatspy & 86.0 & 113.5 & 59.9 & 77.7 & 72.0 \\\hline
Residuals & \scipy & 121.5 & 98.6 & 108.2 & 90.2 & 105.9 \\
 & \astroml & 108.2 & 121.8 & 105.8 & 90.1 & \textbf{83.4} \\
 & \gatspy & 108.2 & 121.8 & 105.8 & 90.1 & \textbf{83.4} \\\hline
\end{tabular}}
\caption{This table shows the 5 highest peaks from the periodograms for Skewness with various treatments of the Full
  Set of data to March 2016. Highlighted in bold are periods close to 82.6 days and in italics periods close
  to 41.3 days.}
\protect\label{table:fullskewtaball}
\end{table}

\begin{table}[!htbp]
\centering
\scalebox{0.75}{
\begin{tabular}{|l|l|r|r|r|r|r|}
\hline
\textbf{Treatment}&\textbf{Routine}&\textbf{Peak 1}&\textbf{Peak 2}&\textbf{Peak 3}&\textbf{Peak 4}&\textbf{Peak 5}\\\hline
None & \scipy & 122.8 & 86.8 & \textbf{83.1} & 114.4 & 79.4 \\
 & \astroml & 122.9 & 114.6 & 86.8 & 86.0 & \textbf{83.0} \\
 & \gatspy & 122.9 & 114.6 & 86.8 & 86.0 & \textbf{83.0} \\\hline
Clipped 1$\sigma$ & \scipy & 122.9 & 86.9 & 114.5 & 65.1 & \textbf{83.1} \\
 & \astroml & 114.7 & 86.9 & 122.9 & 86.1 & 59.9 \\
 & \gatspy & 114.7 & 86.9 & 122.9 & 86.1 & 59.9 \\\hline
Clipped & \scipy & 122.8 & 86.9 & 65.1 & 114.4 & \textbf{83.1} \\
To 3.8 EW & \astroml & 114.6 & 86.9 & 86.1 & 122.8 & 59.9 \\
 & \gatspy & 114.6 & 86.9 & 86.1 & 122.8 & 59.9 \\\hline
Binned & \scipy & 114.6 & 121.9 & 107.5 & 86.8 & 101.3 \\
1 day & \astroml & 114.6 & 107.5 & 121.9 & 86.8 & 101.4 \\
 & \gatspy & 114.6 & 107.5 & 121.9 & 86.8 & 101.4 \\\hline
Binned & \scipy & 114.6 & 86.8 & 122.9 & 107.6 & 79.4 \\
0.5 day & \astroml & 114.6 & 86.8 & 122.9 & 107.6 & 79.4 \\
 & \gatspy & 114.6 & 86.8 & 122.9 & 107.6 & 79.4 \\\hline
Clipped & \scipy & 114.8 & 107.6 & 122.4 & 86.9 & 59.8 \\
1$\sigma$ & \astroml & 114.8 & 107.6 & 122.4 & 86.8 & 59.8 \\
binned 1 day & \gatspy & 114.8 & 107.6 & 122.4 & 86.8 & 59.8 \\\hline
Clipped & \scipy & 114.7 & 86.9 & 122.9 & 59.9 & 107.6 \\
1$\sigma$ & \astroml & 114.7 & 86.8 & 122.9 & 107.6 & 59.9 \\
binned 0.5 day & \gatspy & 114.7 & 86.8 & 122.9 & 107.6 & 59.9 \\\hline
Clipped & \scipy & 114.7 & 107.6 & 122.3 & 86.8 & 59.8 \\
To 3.8 EW & \astroml & 114.7 & 107.6 & 122.3 & 86.8 & 59.8 \\
binned 1 day & \gatspy & 114.7 & 107.6 & 122.3 & 86.8 & 59.8 \\\hline
Clipped & \scipy & 114.6 & 86.9 & 59.9 & 122.8 & 107.5 \\
To 3.8 EW & \astroml & 114.7 & 86.8 & 122.9 & 107.5 & 59.9 \\
binned 0.5 day & \gatspy & 114.7 & 86.8 & 122.9 & 107.5 & 59.9 \\\hline
Residuals & \scipy & 106.9 & 51.4 & 85.4 & \textbf{82.6} & 122.2 \\
 & \astroml & 85.5 & 51.4 & 106.9 & \textbf{82.5} & 60.0 \\
 & \gatspy & 85.5 & 51.4 & 106.9 & \textbf{82.5} & 60.0 \\\hline
\end{tabular}}
\caption{This table shows the 5 highest peaks from the periodograms for Kurtosis with various treatments of the Full
  Set of data to March 2016. Highlighted in bold are periods close to 82.6 days and in italics periods close
  to 41.3 days.}
\protect\label{table:fullkurttaball}
\end{table}
\clearpage

\section{TiO line measurments}
\protect\label{section:apptiotab}

For comparison with the measurements on {\ha}, some similar calculations were done with the TiO peak highlighted in
Fig. \ref{fig:harpsfirstha}, right panel. This is shown in larger scale in Fig. \ref{fig:tispec}.

\begin{table}[!htbp]
\centering
\scalebox{0.75}{
\begin{tabular}{|l|l|r|r|r|r|r|}
\hline
\textbf{Treatment}&\textbf{Routine}&\textbf{Peak 1}&\textbf{Peak 2}&\textbf{Peak 3}&\textbf{Peak 4}&\textbf{Peak 5}\\\hline
Original data & \scipy & 104.4 & 71.0 & 55.5 & 57.4 & 73.5 \\
EW & \astroml & 71.0 & 68.1 & 89.0 & 104.8 & 43.5 \\
 & \gatspy & 68.1 & 71.0 & 104.5 & 105.4 & 75.0 \\\hline
Original data & \scipy & 70.9 & 49.6 & 67.9 & 89.1 & 104.1 \\
EW & \astroml & 71.0 & 51.2 & 76.3 & 49.8 & 89.1 \\
Binned & \gatspy & 70.9 & 49.6 & 67.9 & 104.2 & 89.1 \\\hline
Original data & \scipy & 94.2 & 88.6 & 105.2 & 99.5 & 42.3 \\
PR & \astroml & 88.1 & 76.0 & 105.2 & 99.9 & 121.4 \\
 & \gatspy & 88.2 & 88.6 & 105.9 & 71.2 & 94.0 \\\hline
Original data & \scipy & 74.0 & 99.3 & 71.9 & 68.9 & 92.6 \\
PR & \astroml & 76.1 & 57.9 & 88.7 & 53.3 & 95.0 \\
binned & \gatspy & 99.1 & 73.8 & 71.8 & 120.1 & 121.8 \\\hline
Full data & \scipy & 97.6 & 106.2 & 109.5 & 44.3 & 70.8 \\
EW & \astroml & 42.4 & 40.8 & 46.1 & 54.1 & 50.9 \\
 & \gatspy & 106.2 & 46.0 & 97.1 & 44.2 & 75.0 \\\hline
Full data & \scipy & 121.6 & 43.6 & 40.7 & 49.5 & 108.9 \\
EW & \astroml & 47.0 & 54.2 & 49.8 & 59.4 & 43.7 \\
Binned & \gatspy & 121.6 & 43.6 & 40.7 & 49.5 & 109.0 \\\hline
Full data & \scipy & 50.6 & 42.2 & 68.0 & 106.3 & 46.1 \\
PR & \astroml & 54.1 & 46.1 & 51.6 & 60.2 & 47.0 \\
 & \gatspy & 46.1 & 68.0 & 63.1 & 50.5 & 42.3 \\\hline
Full data & \scipy & 98.2 & 121.7 & 107.6 & 106.3 & 120.4 \\
PR & \astroml & 65.8 & 47.0 & 54.1 & 71.4 & 76.6 \\
binned & \gatspy & 121.0 & 98.5 & 107.6 & 106.4 & 44.5 \\\hline
\end{tabular}}
\caption{This table shows the 5 highest peaks from the periodograms for various treatments of the TiO peak highlighted
  in Fig. \ref{fig:tispec} for Original and Full Sets of {\harps} data.}
\protect\label{table:tipeakall}
\end{table}

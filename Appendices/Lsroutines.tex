\chapter{Lomb-Scargle Routines} % Main appendix title
\protect\label{chapter:lsroutines}
\lhead{Appendix \ref{chapter:lsroutines}. \emph{Lomb-Scargle Routines}}

A selection of Lomb-Scargle routines were employed to process the data in this {\paperorthesis} as no single routine
provided all the facilities needed. As many of the processes undertaken were repetitive in order to produce the tables
for example in Appendix \ref{chapter:pgramdetail}, routines which could operate in ``batch'' mode were required. This
ruled out using Systemic\footnote{Obtainable from http://www.stefanom.org}, for example, which has a GUI interface.

\section{Numerical Recipes}

The Lomb-Scargle routine in the {\numrecs} routines\footnote{This can be obtained from http://numerical-recipes/
  in C and Fortran} was valuable in that it generates False Alarm Probabilities. The routine was enhanced for this
{\paperorthesis} to extend to all the peaks found rather than just the maximum peak.

The main benefit with {\numrecs} is that with this routine clear False Alarm Probabilities may be calculated, but a
drawback is that there is comparatively little control over the periods examined. Some tuning of the oversampling factor
and the maximum multiple of the Nyquist frequency may enable periods of interest to be recovered, but for many of the
spectrographic datasets studied in this {\paperorthesis} such as the {\harps} ones in Section \ref{section:harpsper}
there are dominant apparent periods of 2 days or less which swamp the results and which the routine homes in
upon. Sometimes a careful choice of binning can improve this, but at the cost of losing too many data points.

However in the case of the photometric results in Chapter \ref{chapter:photometry} the predominant peaks, found by this
routine, are the ones of interest and a good estimate of the FAP is returned.

\section{Scipy, AstroML, Gatspy}

Also used were the Python Lomb-Scargle routines provided by {\scipy} library \citep{jones01}, the {\astroml} library,
\citep{vanderplas12} and the {\gatspy} library, \citep{vanderplas15} and incorporated into Python software.

The main benefit of these routines is that the periods to be considered may be specified explicitly. However the
drawback is no FAP is calculated.

If was found that with the photometric results in Chapter \ref{chapter:photometry} where the {\numrecs} routines gave a
clear result and low FAP found, these routines returned identical results, but in the cases where the {\numrecs}
routines did not find the periods sought, such as for the photometric and modelling data, these routines all gave
differing results. In a few cases the {\scipy} library version gave division-by-zero errors and similar, hence the other
two routines were generally preferred. However when the {\scipy} library did work, the results tended to agree more with
the {\gatspy} routine than the {\astroml} routine, so in most cases the {\gatspy} routine was used.

\section{Programs and usage}

The {\numrecs} program takes 5 arguments.

\begin{enumerate}

\item The input file which should be 2-column rows of time and intensity. The program \textbf{Colselect} may be used to
  select the required columns from a multi-column file.

\item The output file in 3 columns, period, power and FAP.

\item The oversampling factor, a float.

\item The highest multiple of the Nyquist frequency, a float.

\item A scaling factor, usually 1.0, which may be applied increase or reduce the differences between the times and then
  adjust the results. In some cases this can cause better results to be produced.

\end{enumerate}

The {\scipy}, {\astroml} and {\gatspy} programs, \textbf{Lspfind}, \textbf{Alspfind} and \textbf{Glspfind} take a
variety of keyword arguments of which the most important are:

\begin{description}

\item[\tt{--period}] Specifies the period range to search in the form start:interval:stop, for example
  \texttt{40d:.01d:130d}. The values are specified as a floating-point number followed by \texttt{d} to denote days or \texttt{h} to
  denote hours. If this is not specified, then the environment variable \texttt{PERIODS} is interrogated for a string of
  similar format.

\item[\tt{--outfile}] Specifies the output file as the three column format as for the {\numrecs} routine.

\item[\tt{--icol}] Specifies the column of the input file, numbered from zero, to use for the intensity value, defaulting to
  1.

\item[\tt{--tcol}] Specifies the column of the input file, numbered from zero, to use for the times, defaulting to 0.

\end{description}

A final argument specifies the input file.

Alternative versions of the programs, \textbf{Lsconv}, \textbf{Alsconv} and \textbf{Glsconv} are provided for quickly
processing Equivalent Width result files described in \ref{section:ewfmt}, taking the argument \texttt{--type} with value \texttt{EW} or \texttt{PR} to
select the appropriate column\footnote{Together with \texttt{PS} to select the now unused column.}.

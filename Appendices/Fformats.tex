\chapter{File formats} % Main appendix title
\protect\label{chapter:fformats}
\lhead{Appendix \ref{chapter:fformats}. \emph{File Formats}}

The software developed uses the following file formats.

\section{Spectral Information File}
\protect\label{section:specinfofmt}

The spectral information file, \texttt{specinfo}, is an XML file used to record the directory in which spectral
observations are held and collate continuum normalisation, details of ranges, etc.

A selection of software was developed to manipulate these files, including in particular \textbf{Sdadmin}, which
provides for interactive adjustment of the ranges.

\section{Equivalent Width Files}
\protect\label{section:ewfmt}

The software developed for processing Equivalent Widths and Peak Ratios uses an 8-column file format with the following
fields in each row.

\begin{enumerate}

\item Observation date, the Julian Date of the observation.

\item Barycentric Julian date of the observation. (In the case of models, this is a duplicate of column 1).

\item Equivalent Width, as calculated.

\item Calculated error on Equivalent Width.

\item (Now unused) a former measure, abandoned, called ``Peak Size''.

\item Error on previous column.

\item Peak Ratio as calculated.

\item Calculated error on Peak Ratio.

\end{enumerate}
 
``Short-cut'' options to some of the software take arguments \texttt{--type ew} or \texttt{--type pr} to select the
columns for Equivalent Width or Peak Ratio respectively\footnote{Plus \texttt{--type ps} to select the now unused column}.